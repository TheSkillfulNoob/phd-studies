\documentclass{article}
\usepackage{minted}
\usepackage{
    amsmath, % mathematical eq and symbols
    graphicx, % manipulate graphics (images)
    xcolor, % color options
    colortbl, % colours to tables
    soul, % highlighting
    multirow, % merging cells in tables
    bm, % bolding variables
    arydshln, % dashed lines
    subcaption, % 2 figures on the same row
    textcomp, gensymb, % degree symbol
    amsfonts, % real number bold
    booktabs, % formal tables
    scrextend, % block Indents
    blkarray, % transitionary probability matrices
    tikz, pgfplots, % coordinate/ fig plots
    ifthen, % logic statements
    CJKutf8, % Chinese characters
    hyperref,
    wrapfig, % For positioning figures
    CJKutf8, % Chinese characters
    ntheorem,
    enumitem, % for bullet spacing
    amsmath,
    ulem,
    centernot,
    pgfcore,
    tabularx,
    stackengine, % for table row height
    bbm % for indicator functions/ bolding digits
}

\usepackage[most]{tcolorbox}
\usepackage{tikz-cd}
\usepackage[margin=0.5in]{geometry}
\usepackage[parfill]{parskip}

% Colors
\definecolor{customtableheaderblue}{HTML}{AAAAEE}
\definecolor{customblue}{RGB}{224, 255, 255}
\definecolor{custompink}{RGB}{255, 204, 229}
\definecolor{customgreen}{RGB}{224, 255, 239}
\definecolor{displaygreen}{RGB}{34, 177, 76}
\definecolor{customorange}{RGB}{255, 239, 224}
\definecolor{custompurple}{RGB}{239, 224, 255}
\definecolor{customgrey}{HTML}{DDDDDD}
\definecolor{customlightgrey}{HTML}{EEEEEE}

\everymath{\displaystyle}

\makeatletter
\newtheoremstyle{MyNonumberplain}%
  {\item[\theorem@headerfont\hskip\labelsep ##1\theorem@separator]}%
  {\item[\theorem@headerfont\hskip\labelsep ##3\theorem@separator]}%
\makeatother
\theoremstyle{MyNonumberplain}
\theorembodyfont{\upshape}

\theoremstyle{break}
\newtheorem*{proof}{Proof. }

\newcommand{\tmmathbf}[1]{\ensuremath{\boldsymbol{#1}}}
\newcommand{\R}{\mathbb{R}}
\newcommand{\Q}{\mathbb{Q}}
\newcommand{\Z}{\mathbb{Z}}
\newcommand{\N}{\mathbb{N}}
\newcommand{\C}{\mathbb{C}}
\newcommand{\cyclic}[1]{\langle #1 \rangle}
\newcommand{\nline}{\begin{tabular}{ll}&\\\end{tabular}}
\newcommand{\nin}{\not\in}
\newcommand{\p}{\phi}
\newcommand{\infixor}{\text{ or }}
\newcommand{\infixand}{\text{ and }}
\newcommand{\ord}[1]{\text{ord}(#1)}
\newcommand{\tmop}{\text}
\newcommand{\xequal}[1]{\stackrel{#1}{=}}
\newcommand{\tmscript}[1]{\text{\scriptsize{$#1$}}}

\newtcolorbox{hintbox}{colback=violet!10,colframe=violet!70,boxrule=0pt,arc=0pt,boxsep=2pt,left=2pt,right=2pt,leftrule=2.5pt}

\newtcolorbox{prfbox}{colback=gray!10,colframe=black!70,boxrule=0pt,arc=0pt,boxsep=2pt,left=2pt,right=2pt,leftrule=0pt}
\newtcolorbox{thmbox}{colback=orange!25,colframe=orange!85,boxrule=0pt,arc=0pt,boxsep=2pt,left=2pt,right=2pt,leftrule=2.5pt}
\newtcolorbox{defbox}{colback=blue!5,colframe=blue!70,boxrule=0pt,arc=0pt,boxsep=2pt,left=2pt,right=2pt,leftrule=2.5pt}
\newtcolorbox{ansbox}{colback=gray!10,colframe=black!70,boxrule=0pt,arc=0pt,boxsep=2pt,left=2pt,right=2pt,leftrule=0pt}
\newtcolorbox{expbox}{colback=green!10,colframe=green!70,boxrule=0pt,arc=0pt,boxsep=2pt,left=2pt,right=2pt,leftrule=2.5pt}
\newtcolorbox{warnbox}{colback=red!15,colframe=red!70,boxrule=0pt,arc=0pt,boxsep=2pt,left=2pt,right=2pt,leftrule=2.5pt}

\newtheorem{warning}{Warning}[section]

\theoremstyle{break}
\newtheorem{theorem}{Theorem}[section]
\newtheorem{corollary}{Corollary}[theorem]
\newtheorem{proposition}{Proposition}[section]
\newtheorem{example}{
    Example
    }[section]
\newtheorem{lemma}[theorem]{Lemma}

\newtheorem{definition}{
Definition}[section]

\newcommand\highlight[2][]{\tikz[overlay]\node[fill=cyan,inner sep=2pt, anchor=text, rectangle, rounded corners=1mm,#1] {#2};\phantom{#2}}

\renewcommand\label[1]{}
\pgfplotsset{compat = 1.18}
\usetikzlibrary{matrix}

% Colors
\definecolor{customtableheaderblue}{HTML}{AAAAEE}
\definecolor{customblue}{RGB}{224, 255, 255}
\definecolor{custompink}{RGB}{255, 204, 229}
\definecolor{customgreen}{RGB}{224, 255, 239}
\definecolor{custompurple}{RGB}{239, 224, 255}

\newcolumntype{M}[1]{>{\centering\arraybackslash}m{#1}}

\tcbuselibrary{theorems}

\newtcolorbox[auto counter]{explanation}[1][]{
title={\bfseries Explanation},
enhanced,
drop shadow={black!50!white},
boxed title style = {
top = 0.1in,
bottom = 0.1in,
size = small,
colback = custompink,
},
coltitle=black,
attach boxed title to top left = {
xshift=-\tcboxedtitlewidth/8,
yshift=-\tcboxedtitleheight/2
},
top = 0.15in,
colback = customgreen,
#1}

\newtcolorbox[auto counter]{formula}[1][]{
title={\bfseries Formula},
enhanced,
drop shadow={black!50!white},
boxed title style = {
top = 0.1in,
bottom = 0.1in,
size = small,
colback = custompink,
},
coltitle=black,
attach boxed title to top left = {
xshift=-\tcboxedtitlewidth/6,
yshift=-\tcboxedtitleheight/2
},
top = 0.15in,
colback = customblue,
#1}

\newcounter{question}
\newtcolorbox[auto counter, number within=subsection]{question}[1][]{
title={\bfseries Question},
enhanced,
drop shadow={black!50!white},
boxed title style = {
top = 0.1in,
bottom = 0.1in,
size = small,
colback = custompink,
},
coltitle=black,
attach boxed title to top left = {
xshift=-\tcboxedtitlewidth/10,
yshift=-\tcboxedtitleheight/2
},
top = 0.15in,
colback = custompurple,
#1}

\setlength{\parindent}{0pt}
\setlist[itemize]{itemsep=0.4em}
\renewcommand{\baselinestretch}{1.15}

\newtcolorbox[auto counter]{code}[1][]{
title={\bfseries Code},
enhanced,
drop shadow={black!50!white},
boxed title style = {
top = 0.1in,
bottom = 0.1in,
size = small,
colback = custompink,
},
coltitle=black,
attach boxed title to top left = {
xshift=-\tcboxedtitlewidth/5,
yshift=-\tcboxedtitleheight/2
},
top = 0.15in,
colback = customorange,
#1}

% Controlling row height
\newcommand\xrowht[2][0]{\addstackgap[.5\dimexpr#2\relax]{\vphantom{#1}}}
\everymath{\displaystyle}
\title{Stanford CS234: Lectures 4 to 6}
\begin{document}

\section{Lecture 4: Model Free Control and Function Approximation}

\section{Lecture 5: Policy Gradient and Search}
Policy gradient/ search is influential in NLP/ Proximal Policy Optimization (training GPT). The core idea:

\begin{defbox}
    \subsubsection*{Intuition of Gradient Research}
    Approximate $V^{\pi}(s) \approx V_{W}(s)$ and $Q_w(s, a) \approx Q^{\pi}(s, a)$ by adjusting weight $w$.
    \\\textbf{Policy} gradient: rather than generating policy from value ($\epsilon$-greedy), directly parametrize policy with $\theta$, i.e.
    \begin{center}
        $\pi_{\theta} (s, a) = \mathbb{P}[a | s; \theta]$: optimize $V(\theta)$ to find policy $\pi$
    \end{center}
\end{defbox}

The brief classification of policy gradient is as follows:
\begin{center}
    \begin{tabular}{|c||c|c|c|}
    \hline
    &  & Value-based & Policy-based & Actor-critic \\ \hline
    Value function & learned & not present & learned \\ \hline
    Policy & implicit ($\epsilon$-greedy) learned & learned\\ \hline
    \end{tabular}
\end{center}

Instead of deterministic/ $\epsilon$-greedy policies, need to focus heavily on \textbf{stochastic} for direct policy search!
\begin{itemize}
\item Repeated Trials, e.g. In rock paper scissors (of many rounds), deterministic policy is easily exploited by adversary.
\item Boundary Condition, e.g. In gridworld, bound to only move one direction (else get stuck/ traverse for long time for slow convergence).
\end{itemize}

In short, poliy objective functions have the following intuition:
\begin{defbox}
    \subsubsection*{Policy Objective Summary}
    \begin{itemize}
    \item Goal: Given policy $\pi_{\theta}(s, a)$, find best parameter $\theta$.
        \\Inherently, an optimization of $V(s_0, \theta)$.
    \item Purpose: Measure quality for policy $\pi_{\theta}$ with policy value at start state $s_0$.
    \item Works for: both episodic/ continuing and infinite horizons.
    \end{itemize}
\end{defbox}

\subsection{Gradient Free Policy Optimization}
They are great simple baselines.
\begin{itemize}
\item Examples: Hill Climbing, Genetic Algo (evolution strategies, cross-entropy method, covariance matrix adaption)
\item Known for decades but embarrassingly well: rivals standard RL techniques!
\item Advantages: Flexible for any policy parameterization, easily to parallelize
    \\Disadvantage: Less sample efficient (ignores temporal structure)
\end{itemize}

\subsection{Policy Gradient}
This section focuses on gradient descent; other popular algos include conjugate gradient and quasi-newton methods.
\\Usually assume \textbf{Episodic MDPs} for easy extension of objectives. The method:
\begin{thmbox}
    Define $V(\theta) = V(s_0, \theta)$, i.e. the value fucntion depending on policy parameters. Then:
    \begin{itemize}
    \item Search the local maximum in $V(s_0, \theta)$ with gradient increments:
        \begin{equation*}
            \Delta \theta = \alpha \nabla_{\theta} V(s_0, \theta) = \alpha
            \begin{pmatrix}
                \frac{\partial V(s_0, \theta)}{\partial \theta_{1}} \\
                \vdots \\
                \frac{\partial V(s_0, \theta)}{\partial \theta_{n}}
            \end{pmatrix}
        \end{equation*}
    \item Assumption: $\pi_{\theta}$ differentiable (and known gradient $\nabla_{\theta} \pi_{\theta}(s, a)$)
    \item We can rewrite policy value $V(s_0, \theta)$ in the following ways:
        \begin{enumerate}
        \item \textbf{Visited States and Actions}: 
            $\mathbb{E}_{\pi_{\theta}} \left[ \sum_{t = 0}^{T} R(s_t, a_t); \pi_{\theta}, s_0 \right]$
        \item \textbf{Weighted Average of Q-values by Actions}:
            $\sum_{a} \pi_{\theta} (a | s_0) Q(s_0, a, \theta)$
        \item \textbf{Trajectories Sampled using $\pi_{\theta}$}: 
            $\sum_{\tau} P(\tau | \theta) R(\tau)$
        \end{enumerate}
    \end{itemize}
\end{thmbox}

In particular, it is of interest to consider writing $V(s_0, \theta)$ in trajectory form:
\begin{prfbox}
    To find the best policy parameter $\theta$, we consider
    \begin{equation*}
        \mathop{\arg\max}\limits_{\theta} V(\theta)
        =
        \mathop{\arg\max}\limits_{\theta} \sum_{\tau} P(\tau ; \theta) R(\tau)
    \end{equation*}
    Taking gradient,
    \begin{aligned}
        \nabla_{\theta} V(\theta) =
        & \nabla_{\theta} \sum_{\tau} P(\tau; \theta) R(\tau) \\
        & \sum_{\tau} \nabla_{\theta} P(\tau; \theta) R(\tau) \text{($R$ being indep of $\theta$)}\\
        & \sum_{\tau} \frac{P(\tau; \theta)}{P(\tau; \theta)} \nabla_{\theta} P(\tau; \theta) R(\tau) \\
        & \sum_{\tau} R(\tau) P(\tau; \theta) \nabla_{\theta} \textcolor{orange}{\log P(\tau; \theta)} ~~~~\text{(log-likelihood)} \\
    \end{aligned}

    \textbf{Approximate} in practice using $m$ sample trajectories under $\pi_{\theta}$:
    \begin{equation*}
        \nabla_{\theta} V(\theta) \approx \hat{g} = \frac{1}{m} \sum_{i = 1}^{m} R(\tau^{(i)}) \nabla_{\theta} \log P(\tau^{(i)}, \theta)
    \end{equation*}
\end{prfbox}

But trajectories can be decomposed into states and actions:
\begin{prfbox}
    \begin{aligned}
        \nabla_{\theta} \log P(\tau^{(i)}; \theta) =
        & \nabla_{\theta} \log \left[ \textcolor{cyan}{\mu(s_0)} 
            \textcolor{magenta}{\prod_{t = 0}^{T-1} \pi_{\theta} (a_t | s_t)}
            \textcolor{orange}{P(s_{t+1} | a_{t+1}, s_{0:t}, a_{0:t})} \right] \\
        & \textcolor{magenta}{\sum_{\tau}} \nabla_{\theta} \log \textcolor{magenta}{\pi_{\theta} (a_t | s_t)} \\
    \end{aligned}
\end{prfbox}

Here 
    \begin{itemize}
    \item We call $\sum_{\tau} \nabla_{\theta} \log \pi_{\theta} (a_t | s_t)$ the \textbf{score function}.
    \item \textcolor{cyan}{the initial state $\mu(s_0)$ is constant};  
        \textcolor{orange}{dynamics model $P(s_{t+1} | a_{t+1}, s_{0:t}, a_{0:t})$ is invariant to $\theta$.}
    \item In other words, no dynamics model is required to approximate the policy parameter $\theta$.
    \end{itemize}

\begin{hintbox}
    \textbf{Questions}
    \begin{enumerate}
    \item Why trajectory form is practical ("better in training")?
    \item Why is log-likelihood ratio important here? What does it enable?
    \end{enumerate}
\end{hintbox}

\subsection{Selecting a Right Policy}

\subsubsection{Softmax Policy}
\begin{itemize}
\item In softmax, \textbf{exponentially weight} quantities of linear combination of features as probabilities (that add to 1):
    \begin{equation*}
        \pi_{\theta}(s, a) = \frac{e^{\phi (s, a)^{T} \theta}}{\sum_{a} e^{\phi (s, a)^{T} \theta}}
    \end{equation*}
\item Then the score function can be written as $\nabla_{\theta}(s, a) = \phi(s, a) - \mathbb{E}_{\pi_{\theta} \[ \phi(s, \cdot)\]}$
\end{itemize}

\subsubsection{Gaussian Policy}
\begin{itemize}
\item A normal distribution is natural for continuous action spaces; at times used by deep NN.
\item Action $a \sim N(\mu(s), \sigma^2)$. Mean $\mu(s) = \textcolor{cyan}{\phi(s)}^{T} \textcolor{orange}{\theta}$ is a \textcolor{orange}{linear combination} of \textcolor{cyan}{state features}. 
\item Then the score function can be written as $\nabla_{\theta}(s, a) = \phi(s, a) - \mathbb{E}_{\pi_{\theta}} \left[ \phi(s, \cdot)\right]$
\end{itemize}

When to use policy gradient?
\begin{enumerate}
\item Differentiable reward functions
\item No dynamics required
\item Useful for both infinite horizon and episodic settings
\item Intuition: $R(\tau^{(i)})$ is replacable by other functions that \textit{measures the wellness of sample $x$}
    \\Essentially, moving in the direction of $\hat{g_{i}} = f(x_i) \nabla_{\theta} \textcolor{orange}{\log p(x_i | \theta)}$ pushes up the \textcolor{orange}{log probability} proportionally.
\end{enumerate}

\begin{hintbox}
    What are the purposes of selecting Softmax and Gaussian policies?
\end{hintbox}

The generalization is as follows:
\begin{thmbox}
    \subsubsection*{Policy Gradient Theorem}
    Assumption: Differentiable Policy $\pi_{\theta}(s, a)$, objective $J = $
    \begin{aligned}
        J_1 & \text{(Episodic)} \\
        J_{avR} & \text{(Avg. Reward over time)} \\
        \frac{1}{1 - \gamma} J_{avV} & \text{(Avg. Value over time)} \\
    \end{aligned}
    
    In any case, the policy gradient is 
    $\nabla_{\theta} J(\theta) = \mathbb{E}_{\pi_{\theta}} \left[ \nabla_{\theta} \log \pi_{\theta} (s, a) Q^{\pi_{\theta}} (s, a)\right]$
\end{thmbox}

\subsection{VFA under TDL}


\section{Lecture 5-6: Remedies to Reduce Variance in Policy Gradient}

\end{document}