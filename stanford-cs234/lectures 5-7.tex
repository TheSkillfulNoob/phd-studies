\documentclass{article}
\usepackage{minted}
\usepackage{
    amsmath, % mathematical eq and symbols
    graphicx, % manipulate graphics (images)
    xcolor, % color options
    colortbl, % colours to tables
    soul, % highlighting
    multirow, % merging cells in tables
    bm, % bolding variables
    arydshln, % dashed lines
    subcaption, % 2 figures on the same row
    textcomp, gensymb, % degree symbol
    amsfonts, % real number bold
    booktabs, % formal tables
    scrextend, % block Indents
    blkarray, % transitionary probability matrices
    tikz, pgfplots, % coordinate/ fig plots
    ifthen, % logic statements
    CJKutf8, % Chinese characters
    hyperref,
    wrapfig, % For positioning figures
    CJKutf8, % Chinese characters
    ntheorem,
    enumitem, % for bullet spacing
    amsmath,
    ulem,
    centernot,
    pgfcore,
    tabularx,
    stackengine, % for table row height
    bbm % for indicator functions/ bolding digits
}

\usepackage[most]{tcolorbox}
\usepackage{tikz-cd}
\usepackage[margin=0.5in]{geometry}
\usepackage[parfill]{parskip}

% Colors
\definecolor{customtableheaderblue}{HTML}{AAAAEE}
\definecolor{customblue}{RGB}{224, 255, 255}
\definecolor{custompink}{RGB}{255, 204, 229}
\definecolor{customgreen}{RGB}{224, 255, 239}
\definecolor{displaygreen}{RGB}{34, 177, 76}
\definecolor{customorange}{RGB}{255, 239, 224}
\definecolor{custompurple}{RGB}{239, 224, 255}
\definecolor{customgrey}{HTML}{DDDDDD}
\definecolor{customlightgrey}{HTML}{EEEEEE}

\everymath{\displaystyle}

\makeatletter
\newtheoremstyle{MyNonumberplain}%
  {\item[\theorem@headerfont\hskip\labelsep ##1\theorem@separator]}%
  {\item[\theorem@headerfont\hskip\labelsep ##3\theorem@separator]}%
\makeatother
\theoremstyle{MyNonumberplain}
\theorembodyfont{\upshape}

\theoremstyle{break}
\newtheorem*{proof}{Proof. }

\newcommand{\tmmathbf}[1]{\ensuremath{\boldsymbol{#1}}}
\newcommand{\R}{\mathbb{R}}
\newcommand{\Q}{\mathbb{Q}}
\newcommand{\Z}{\mathbb{Z}}
\newcommand{\N}{\mathbb{N}}
\newcommand{\C}{\mathbb{C}}
\newcommand{\cyclic}[1]{\langle #1 \rangle}
\newcommand{\nline}{\begin{tabular}{ll}&\\\end{tabular}}
\newcommand{\nin}{\not\in}
\newcommand{\p}{\phi}
\newcommand{\infixor}{\text{ or }}
\newcommand{\infixand}{\text{ and }}
\newcommand{\ord}[1]{\text{ord}(#1)}
\newcommand{\tmop}{\text}
\newcommand{\xequal}[1]{\stackrel{#1}{=}}
\newcommand{\tmscript}[1]{\text{\scriptsize{$#1$}}}

\newtcolorbox{hintbox}{colback=violet!10,colframe=violet!70,boxrule=0pt,arc=0pt,boxsep=2pt,left=2pt,right=2pt,leftrule=2.5pt}

\newtcolorbox{prfbox}{colback=gray!10,colframe=black!70,boxrule=0pt,arc=0pt,boxsep=2pt,left=2pt,right=2pt,leftrule=0pt}
\newtcolorbox{thmbox}{colback=orange!25,colframe=orange!85,boxrule=0pt,arc=0pt,boxsep=2pt,left=2pt,right=2pt,leftrule=2.5pt}
\newtcolorbox{defbox}{colback=blue!5,colframe=blue!70,boxrule=0pt,arc=0pt,boxsep=2pt,left=2pt,right=2pt,leftrule=2.5pt}
\newtcolorbox{ansbox}{colback=gray!10,colframe=black!70,boxrule=0pt,arc=0pt,boxsep=2pt,left=2pt,right=2pt,leftrule=0pt}
\newtcolorbox{expbox}{colback=green!10,colframe=green!70,boxrule=0pt,arc=0pt,boxsep=2pt,left=2pt,right=2pt,leftrule=2.5pt}
\newtcolorbox{warnbox}{colback=red!15,colframe=red!70,boxrule=0pt,arc=0pt,boxsep=2pt,left=2pt,right=2pt,leftrule=2.5pt}

\newtheorem{warning}{Warning}[section]

\theoremstyle{break}
\newtheorem{theorem}{Theorem}[section]
\newtheorem{corollary}{Corollary}[theorem]
\newtheorem{proposition}{Proposition}[section]
\newtheorem{example}{
    Example
    }[section]
\newtheorem{lemma}[theorem]{Lemma}

\newtheorem{definition}{
Definition}[section]

\newcommand\highlight[2][]{\tikz[overlay]\node[fill=cyan,inner sep=2pt, anchor=text, rectangle, rounded corners=1mm,#1] {#2};\phantom{#2}}

\renewcommand\label[1]{}
\pgfplotsset{compat = 1.18}
\usetikzlibrary{matrix}

% Colors
\definecolor{customtableheaderblue}{HTML}{AAAAEE}
\definecolor{customblue}{RGB}{224, 255, 255}
\definecolor{custompink}{RGB}{255, 204, 229}
\definecolor{customgreen}{RGB}{224, 255, 239}
\definecolor{custompurple}{RGB}{239, 224, 255}

\newcolumntype{M}[1]{>{\centering\arraybackslash}m{#1}}

\tcbuselibrary{theorems}

\newtcolorbox[auto counter]{explanation}[1][]{
title={\bfseries Explanation},
enhanced,
drop shadow={black!50!white},
boxed title style = {
top = 0.1in,
bottom = 0.1in,
size = small,
colback = custompink,
},
coltitle=black,
attach boxed title to top left = {
xshift=-\tcboxedtitlewidth/8,
yshift=-\tcboxedtitleheight/2
},
top = 0.15in,
colback = customgreen,
#1}

\newtcolorbox[auto counter]{formula}[1][]{
title={\bfseries Formula},
enhanced,
drop shadow={black!50!white},
boxed title style = {
top = 0.1in,
bottom = 0.1in,
size = small,
colback = custompink,
},
coltitle=black,
attach boxed title to top left = {
xshift=-\tcboxedtitlewidth/6,
yshift=-\tcboxedtitleheight/2
},
top = 0.15in,
colback = customblue,
#1}

\newcounter{question}
\newtcolorbox[auto counter, number within=subsection]{question}[1][]{
title={\bfseries Question},
enhanced,
drop shadow={black!50!white},
boxed title style = {
top = 0.1in,
bottom = 0.1in,
size = small,
colback = custompink,
},
coltitle=black,
attach boxed title to top left = {
xshift=-\tcboxedtitlewidth/10,
yshift=-\tcboxedtitleheight/2
},
top = 0.15in,
colback = custompurple,
#1}

\setlength{\parindent}{0pt}
\setlist[itemize]{itemsep=0.4em}
\renewcommand{\baselinestretch}{1.15}

\newtcolorbox[auto counter]{code}[1][]{
title={\bfseries Code},
enhanced,
drop shadow={black!50!white},
boxed title style = {
top = 0.1in,
bottom = 0.1in,
size = small,
colback = custompink,
},
coltitle=black,
attach boxed title to top left = {
xshift=-\tcboxedtitlewidth/5,
yshift=-\tcboxedtitleheight/2
},
top = 0.15in,
colback = customorange,
#1}

% Controlling row height
\newcommand\xrowht[2][0]{\addstackgap[.5\dimexpr#2\relax]{\vphantom{#1}}}
\everymath{\displaystyle}
\title{Stanford CS234: Lectures 5 to 7}
\begin{document}

Policy gradient/ search is influential in NLP/ Proximal Policy Optimization (training GPT). The core idea:

\section{Why Policy Gradient, not Value Based?}
\begin{defbox}
    \subsubsection*{Intuition of Gradient Search}
    Approximate $V^{\pi}(s) \approx V_{W}(s)$ and $Q_w(s, a) \approx Q^{\pi}(s, a)$ by adjusting weight $w$.
    \\\textbf{Policy} gradient: rather than generating policy from value ($\epsilon$-greedy), directly parametrize policy with $\theta$, i.e.
    \begin{center}
        $\pi_{\theta} (s, a) = \mathbb{P}[a | s; \theta]$: optimize $V(\theta)$ to find policy $\pi$
    \end{center}
\end{defbox}

The brief classification of policy gradient is as follows:
\begin{center}
    \begin{tabular}{|c||c|c|c|}
    \hline
      & Value-based & Policy-based & Actor-critic \\ \hline
    Value function & learned & not present & learned \\ \hline
    Policy & implicit ($\epsilon$-greedy) & learned & learned\\ \hline
    \end{tabular}
\end{center}

Instead of deterministic/ $\epsilon$-greedy policies, need to focus heavily on \textbf{stochastic} for direct policy search!
\begin{itemize}
\item Repeated Trials, e.g. In rock paper scissors (of many rounds), deterministic policy is easily exploited by adversary.
\item Boundary Condition, e.g. In gridworld, bound to only move one direction (else get stuck/ traverse for long time for slow convergence).
\end{itemize}

\subsection{Gradient Free Policy Optimization}
We begin with simple (but great) gradient-free baselines.
\begin{itemize}
\item Examples: Hill Climbing, Genetic Algo (evolution strategies, cross-entropy method, covariance matrix adaption)
\item Known for decades but embarrassingly well: rivals standard RL techniques!
\item Advantages: Flexible for any policy parameterization, easily to parallelize
    \\Disadvantage: Less sample efficient (ignores temporal structure)
\end{itemize}

\section{Main Objective and Log-likelihood Trick for Policy Gradient}

\subsection{Policy Gradient}
This section focuses on gradient descent; other popular algos include conjugate gradient and quasi-newton methods.
\\We assume \textbf{Episodic MDPs} for easy extension of objectives. We first outline the problem as follows:

\begin{defbox}
    \subsubsection*{Policy Objective Summary}
    \begin{itemize}
    \item Goal: Given policy $\pi_{\theta}(s, a)$, find best parameter $\theta$.
        Inherently, an optimization of $V(s_0, \theta)$ (i.e. the value fucntion depending on policy parameters).
    \item Purpose: Measure quality for policy $\pi_{\theta}$ with policy value at start state $s_0$.
    \item Works for: both episodic/ continuing and infinite horizons.
    \end{itemize}
\end{defbox}

The method:
\begin{thmbox}
    \subsubsection*{Vanilla Policy Gradient: Problem Formation}
    \begin{itemize}
    \item Search the local maximum of policy value $V(s_0, \theta)$ with gradient increments:
        \begin{equation*}
            \Delta \theta = \alpha \nabla_{\theta} V(s_0, \theta) = \alpha
            \begin{pmatrix}
                \frac{\partial V(s_0, \theta)}{\partial \theta_{1}} \\
                \vdots \\
                \frac{\partial V(s_0, \theta)}{\partial \theta_{n}}
            \end{pmatrix}
        \end{equation*}
    \item Assumption: $\pi_{\theta}$ differentiable (and known gradient $\nabla_{\theta} \pi_{\theta}(s, a)$)
    \item We can rewrite $V(s_0, \theta)$ in the following ways:
        \begin{enumerate}
        \item \textbf{Visited States and Actions}: 
            $\mathbb{E}_{\pi_{\theta}} \left[ \sum_{t = 0}^{T} R(s_t, a_t); \pi_{\theta}, s_0 \right]$
        \item \textbf{Weighted Average of Q-values by Actions}:
            $\sum_{a} \pi_{\theta} (a | s_0) Q(s_0, a, \theta)$
        \item \textbf{Trajectories Sampled using $\pi_{\theta}$}: 
            $\sum_{\tau} P(\tau | \theta) R(\tau)$
        \end{enumerate}
    \end{itemize}
\end{thmbox}

\subsection{Log-Likelihood Trick and Score}
In particular, it is of interest to consider writing $V(s_0, \theta)$ in trajectory form:
\begin{prfbox}
    To find the best policy parameter $\theta$, we consider
    \begin{equation*}
        \mathop{\arg\max}\limits_{\theta} V(\theta)
        =
        \mathop{\arg\max}\limits_{\theta} \sum_{\tau} P(\tau ; \theta) R(\tau)
    \end{equation*}
    Taking gradient,
    $\begin{aligned}
        \nabla_{\theta} V(\theta) =
        & \nabla_{\theta} \sum_{\tau} P(\tau; \theta) R(\tau) \\
        & \sum_{\tau} \nabla_{\theta} P(\tau; \theta) R(\tau) \text{($R$ being indep of $\theta$)}\\
        & \sum_{\tau} \frac{P(\tau; \theta)}{P(\tau; \theta)} \nabla_{\theta} P(\tau; \theta) R(\tau) \\
        & \sum_{\tau} R(\tau) P(\tau; \theta) \nabla_{\theta} \textcolor{orange}{\log P(\tau; \theta)} ~~~~\text{(log-likelihood)} \\
    \end{aligned}$

    \textbf{Approximate} in practice using $m$ sample trajectories under $\pi_{\theta}$:
    \begin{equation*}
        \nabla_{\theta} V(\theta) \approx \hat{g} = \frac{1}{m} \sum_{i = 1}^{m} R(\tau^{(i)}) \nabla_{\theta} \log P(\tau^{(i)}, \theta)
    \end{equation*}
\end{prfbox}

But trajectories can be decomposed into states and actions:
\begin{prfbox}
    $\begin{aligned}
        \nabla_{\theta} \log P(\tau^{(i)}; \theta) =
        & \nabla_{\theta} \log \left[ \textcolor{cyan}{\mu(s_0)} 
            \textcolor{magenta}{\prod_{t = 0}^{T-1} \pi_{\theta} (a_t | s_t)}
            \textcolor{orange}{P(s_{t+1} | a_{t+1}, s_{0:t}, a_{0:t})} \right] \\
        & =\textcolor{magenta}{\sum_{\tau}} \nabla_{\theta} \log \textcolor{magenta}{\pi_{\theta} (a_t | s_t)} \\
    \end{aligned}$
\end{prfbox}

Here 
    \begin{itemize}
    \item We call $\sum_{\tau} \nabla_{\theta} \log \pi_{\theta} (a_t | s_t)$ the \textbf{score function}.
    \item \textcolor{cyan}{the initial state $\mu(s_0)$ is constant};  
        \textcolor{orange}{dynamics model $P(s_{t+1} | a_{t+1}, s_{0:t}, a_{0:t})$ is invariant to $\theta$.}
    \item In other words, \textcolor{red}{no dynamics model is required} to approximate the policy parameter $\theta$.
    \end{itemize}

\begin{hintbox}
    \textbf{Questions}
    \begin{enumerate}
    \item Why trajectory form is practical ("better in training")?
        \begin{prfbox}
            Two major reasons:
            \begin{enumerate}
            \item \textbf{Flexible}
                \\"Black box access" that only requires \textcolor{blue}{generated trajectory rollouts} without differentiable envt. model.
                \\Allows turning $\nabla_{\theta} P(\tau)$ into $P(\tau) \nabla_{\theta} \log P(\tau)$, i.e. unbiased gradient estimate with simplicity.
            \item \textbf{Easy Implementation}
                \\Transition probabilities $P(s_{t+1} | s_t, a_t)$ don't appear in gradient after log derivative.
                \\Only actions, episodes, states and rewards (not $P$) to compute gradient!
            \end{enumerate}
        \end{prfbox}
    \item Why is log-likelihood ratio important here? What does it enable?
        \begin{prfbox}
            Log trick enables PG without environment back-propagation.
            \\Without model $\mathbf{P}$ required, log likelihood enables additive (instead of multiplicative) decomposition.
        \end{prfbox}
    \end{enumerate}
\end{hintbox}

\subsection{Selecting a Right Policy}

\subsubsection{Softmax Policy}
\begin{itemize}
\item In softmax, \textbf{exponentially weight} quantities of linear combination of features as probabilities (that add to 1):
    \begin{equation*}
        \pi_{\theta}(s, a) = \frac{e^{\phi (s, a)^{T} \theta}}{\sum_{a} e^{\phi (s, a)^{T} \theta}}
    \end{equation*}
\item Then the score function can be written as 
    $\nabla_{\theta} \log \pi_{\theta}(s, a) = \phi(s, a) - \mathbb{E}_{\pi_{\theta} [ \phi(s, \cdot) ]}$
\end{itemize}

\subsubsection{Gaussian Policy}
\begin{itemize}
\item A normal distribution is natural for continuous action spaces; at times used by deep NN.
\item Action $a \sim N(\mu(s), \sigma^2)$. Mean $\mu(s) = \textcolor{cyan}{\phi(s)}^{T} \textcolor{orange}{\theta}$ is a \textcolor{orange}{linear combination} of \textcolor{cyan}{state features}. 
\item Then the score function can be written as $\nabla_{\theta} \log \pi_{\theta}(s, a) = \frac{(a - \mu(s)) \phi(s)}{\sigma^2} $
\end{itemize}

\begin{hintbox}
    What are the purposes of selecting Softmax and Gaussian policies?
    \begin{prfbox}
        Policy classes should allow \textcolor{blue}{(1) Easy action sampling} and \textcolor{blue}{(2) Straightforward gradient $\nabla_{\theta} \log \pi_{\theta}(s, a)$ computation}.
        \\The two policies are both differentiable, allowing (2). For (1):
        \begin{itemize}
        \item \textbf{Softmax}: \textbf{Discrete} action probabilities that returns a nice (simple) gradient form.
        \item \textbf{Gaussian}: \textbf{Continuous} actions featuring straightforward gradient, common in robotics/ continuous control. 
        \end{itemize}
    \end{prfbox}
\end{hintbox}

\subsection{Summary: PG}
A summary of policy gradient:
\begin{defbox}
    \subsubsection*{Intermediate Summary of PG}
    \begin{itemize}
    \item Core idea: $\nabla_{\theta} V(\theta) = \mathbb{E}_{\pi_{\theta}} \left[ \nabla_{\theta} \textcolor{cyan}{\log \pi_{\theta} (s, a)} \textcolor{blue}{Q^{\pi_{\theta}} (s, a)} \right]$
        \\\begin{prfbox}
            Optimize $\mathop{\arg\max}\limits_{\theta} J(\pi_{\theta}) = \mathbb{E}_{\tau \sim \pi_{\theta}} \left[ \sum_{t = 0}^{\infty} \gamma^{t} r_{t}\right]$ with SGD on $\theta$:
            \begin{equation*}
                g = \nabla_{\theta} J(\pi_{\theta}) = \mathbb{E}_{\tau \sim \pi_{\theta}} \left[ \sum_{t = 0}^{\infty} \gamma^{t} \nabla_{\theta} \log \pi_{\theta} (a_t | s_t) A^{\pi_{\theta}} (s_t, a_t) \right]
            \end{equation*}
        \end{prfbox}
    \item State-action pairs with higher $\hat{Q}$ increases probabilities in average
    \item Direction of $\theta$ dependent on gradient of \textcolor{cyan}{$\ln \pi(S_t, A_t, \theta)$} AND \textcolor{blue}{Q-values/ returns}
    \item NOT guaranteed to converge to global optima (just local!)
    \end{itemize}

    \textbf{When to use?}
    \begin{enumerate}
    \item Differentiable reward functions; No dynamics required
    \item Useful for both infinite horizon and episodic settings
    \item Intuition: $R(\tau^{(i)})$ is replacable by other functions that \textit{measures the wellness of sample $x$}
        \\Essentially, moving in the direction of $\hat{g_{i}} = f(x_i) \nabla_{\theta} \textcolor{orange}{\log p(x_i | \theta)}$ pushes up the \textcolor{orange}{log probability} proportionally.
    \end{enumerate}
\end{defbox}

The generalization of PG is as follows:
\begin{thmbox}
    \subsubsection*{Policy Gradient Theorem}
    Assumption: Differentiable Policy $\pi_{\theta}(s, a)$, objective $J = 
    \begin{cases}
        J_1 & \text{(Episodic)} \\
        J_{avR} & \text{(Avg. Reward over time)} \\
        \frac{1}{1 - \gamma} J_{avV} & \text{(Avg. Value over time)} \\
    \end{cases}
    $
    In any case, the policy gradient is 
    $\nabla_{\theta} J(\theta) = \mathbb{E}_{\pi_{\theta}} \left[ \nabla_{\theta} \log \pi_{\theta} (s, a) Q^{\pi_{\theta}} (s, a)\right]$
\end{thmbox}

\subsubsection*{Summary and Improvements of Policy-based RL}
\begin{center}
    \begin{tabular}{|c||c|c|}
    \hline
    Criteria & Advantages & Disadvantages \\ \hline
    Convergence & Better Properties & Typical Local Optimum \\ \hline
    Flexibility & Effective in high-dim./ continuous action spaces & Inefficient, high-variance \\ 
    & Can learn stochastic policies & policy evaluation \\ \hline
    \end{tabular}
\end{center}

\section{Variance Issues and Remedies}

\subsection{Problem: High Variance in PG}
\begin{defbox}
    Currently, use
    \begin{equation*}
        \nabla_{\theta} V(\theta) = \frac{1}{m} \sum_{i = 1}^{m} R(\tau^{(i)}) \sum_{t = 0}^{T-1} \nabla_{\theta} \log \pi_{\theta} (a_t^{(i)}, s_t^{(i)})
    \end{equation*}
    to estimate
    \begin{equation*}
        \nabla_{\theta} \mathbb{E}_{\tau} [R] = \mathbb{E}_{\tau} \left[ \left(\sum_{t = 0}^{T-1} r_{t} \right) \sum_{t=0}^{T-1} \nabla_{\theta} \log \pi_{\theta} (a_t | s_t) \right]
    \end{equation*}
    It is unbiased but \textcolor{red}{noisy (high variance)!}
\end{defbox}
On the high variance, several remedies serve as improvements:

\subsection{Fix 1: Temporal Structure}
\begin{itemize}
\item Focus on \textcolor{blue}{single reward item} at once:
    \begin{equation*}
        \nabla_{\theta} \mathbb{E}[\textcolor{blue}{r_{t'}}] = \mathbb{E}[r_{t'} \sum_{t = 0}^{t'} \nabla_{\theta} \log \pi_{\theta} (a_t | s_t)]
    \end{equation*}
\item \textcolor{orange}{Sum up over $t$} to obtain:
    $\begin{aligned}
        V(\theta) = \nabla_{\theta} \mathbb{E}[\textcolor{orange}{R}] & 
        = \mathbb{E} [ \textcolor{orange}{\sum_{t' = 0}^{T-1} r_{t'}} \sum_{t = 0}^{t'} \nabla_{\theta} \log \pi_{\theta} (a_t | s_t)] \\
        & = \mathbb{E} [ \sum_{t = 0}^{T-1} \nabla_{\theta} \log \pi_{\theta} (a_t , s_t) \textcolor{orange}{\sum_{t' = t}^{T-1} r_{t'}}] \quad \text{(because later decisions don't influence past rewards)}\\
    \end{aligned}$
\item This can be further simplified: trajectory $\tau^{(i)}$ has return \textcolor{orange}{$G_t^{(i)} = \sum_{t' = t}^{T-1} r_{t'}^{(i)}$}. Hence
    \begin{equation*}
        \nabla_{\theta} \mathbb{E}[R] \approx \frac{1}{m} \sum_{i = 1}^{m} \sum_{t = 0}^{T-1} \nabla_{\theta} \log \pi_{\theta} (a_t, s_t) \textcolor{orange}{G_t^{(i)}}
    \end{equation*}
\end{itemize}

\begin{thmbox}
    \subsubsection*{Monte-Carlo Policy Gradient}
    Making use of likelihood ratio / score function and temporal structure, update param $\theta$ with
    \begin{equation*}
        \Delta \theta_{t} = \alpha \nabla_{\theta} \log \pi_{\theta} (s_t, a_t) G_{t}
    \end{equation*}
    after initializing $\theta$ arbitrarily.
\end{thmbox}

\begin{hintbox}
    Q: How does Temporal structure reduce variance?
    \begin{prfbox}
        \begin{itemize}
        \item Each $r_{t'}$ is paired with actions ONLY from time 0 to $t'$.
        \item Actions after $t'$ \textcolor{blue}{does NOT affect $r_{t'}$}:
            \\Assigning rewards to ONLY actions that influence it achieves noise reduction.
        \end{itemize}
    \end{prfbox}
\end{hintbox}

\subsection{Fix 2: Baseline Function}
As iteration costs time and computational resources, we desire \textcolor{blue}{quick convergence} to local optima.

\begin{thmbox}
    \subsubsection*{Baselines: Unbiasedness and Other Considerations}
    $\nabla_{\theta} \mathbb{E}_{\tau} [R] = \mathbb{E}_{\tau} \left[ \left(\sum_{t' = t}^{T-1} r_{t'} \textcolor{magenta}{- b(s_t)} \right) \sum_{t=0}^{T-1} \nabla_{\theta} \log \pi_{\theta} (a_t | s_t) \right]$

    \textbf{Why it works?}
    \begin{itemize}
    \item Unbiased for any $b$ if $b$ is a function of $s$ but not $\theta$, because
        \\$\mathbb{E}_{\tau} \left[ \nabla_{\theta} \log \pi(a_t | s_t; \theta) b(s_t) \right] = 0$
    \item A near-optimal baseline choice is the expected return 
        $b(s_t) \approx \mathbb{E} \left[ \sum_{t' = t}^{T-1} r_{t'} \right]$
    \item Other choices: State-value function $V^{\pi}(s) = \mathbb{E}_{a \sim \pi} \left[ Q^{\pi} (s, a) \right]$
    \end{itemize}
    \textbf{Interpretation}: Increase logprob of action at proportionally to how much returns 
    $\sum^{T-1}_{t'= t} r_{t'}$ are better than expected.
\end{thmbox}

Mathematically, baseline functions achieve variance reduction as follows:
\begin{prfbox}
    \textbf{Core idea: Break down $t$} in the summation:
    \begin{itemize}
    \item As we sample from trajectories $\tau$ (MC),
        $\begin{aligned}
            \text{Var}[ \nabla_{\theta} [R]] & 
            = \text{Var}[ \sum_{t = 0}^{T-1} \nabla_{\theta} \log \pi(a_t | s_t; \theta) (R_t (s_t) - b(s_t))]\\ & 
            \approx \sum_{t = 0}^{T-1} \mathbb{E}_{\tau} \left[ \text{Var}[\nabla_{\theta} \log \pi(a_t|s_t ; \theta) (R_t(s_t) - b(s_t))] \right]\\
        \end{aligned}$
    \item For each $t$, write variance Var$[X]$ as $\mathbb{E}[X^2] - (\mathbb{E}[X])^2$:
        \begin{equation*}
            \mathbb{E}\left[\left((\nabla_{\theta} \log \pi(a_t | s_t ; \theta)) (R_t(s_t) - b(s_t)) \right)^2 \right]
            - \textcolor{cyan}{\mathbb{E}\left[\left((\nabla_{\theta} \log \pi(a_t | s_t ; \theta)) (R_t(s_t) - b(s_t)) \right) \right]^2}
        \end{equation*}
    \item \textcolor{cyan}{Second term} is not affected by choice of $b(s)$ (\textcolor{cyan}{unbiased} = same expectation). 
    \\
        $\begin{aligned}
            \text{The variance equals} & 
            \mathop{\arg\min}\limits_{b} ~~\mathbb{E}\left[\left((\nabla_{\theta} \log \pi(a_t | s_t ; \theta))\right)^2 \left((G_t(s_t) - b(s_t)) \right)^2 \right]\\ 
            = & \mathop{\arg\min}\limits_{b} ~~\mathbb{E}_{s \sim d^{\pi}}\left[ \mathbb{E}_{a \sim \pi(\cdot | s), G | s, a} \left[ \left((\nabla_{\theta} \log \pi(a_t | s ; \theta))\right)^2 \left((G_t(s_t) - b(s_t)) \right)^2 \right] \right]\\
        \end{aligned}$
    \item A \textbf{weighted least-squares} problem that minimizes
        \begin{equation*}
            \sum_{i} \sum_{t} |b(s^{i}_{t}) - \textcolor{blue}{G^{i}_{t}}|^2 \quad\quad \text{(\textcolor{blue}{$G$ (or $A = G - b$)} being target)}
        \end{equation*}            
        with solution (after taking zero gradient)
        \begin{equation*}
            b(s) \approx \mathbb{E}_{a \sim \pi(\cdot | s), G | s, a} [G_t (s)]
        \end{equation*}
    \end{itemize}
\end{prfbox}

\begin{hintbox}
    Q: Intuitively, what are the uses of baseline functions? How does variance get reduced?
    \begin{prfbox}
        \begin{itemize}
        \item Second term: If $b(s)$ is "close" to true $V^{\pi}(s)$ (high correlation between the two), 
            $G_t(s_t) - b(s_t)$ achieves lower variance.
        \item First term: Log-probability only updates strongly (decisively, to increase variance) if significant return differences to baseline is observed.
        \end{itemize}
    \end{prfbox}
    Q: What does it mean by "Near optimal"? When is it not?
    \begin{prfbox}
        \begin{itemize}
        \item According to above, when variance reduction of the second term $>$ variance increase of first term.
            \\Usually the case as $b(s) \approx V^{\pi}(s)$ typically, achieving maximal variance reduction in theory.
        \item If learned baseline is inaccurate (i.e. poor function approximator/ predictor), less reduction.
            \\Extreme cases: Taking $b(s) = \text{const}$ or 0 could be better.
        \end{itemize}
    \end{prfbox}
\end{hintbox}

\subsection{Fix 3: Actor-Critic}
The original $G_{t}^{i}$ estimates expected discounted sum of returns (from single roll): unbiased but \textcolor{red}{high variance}.
To solve this:
\begin{itemize}
\item Leverage \textcolor{cyan}{bootstrapping and approximation} (similar to TD vs MC and VFA) to introduce bias.
\item Use "Critic" to estimate the ratio $\frac{V}{Q}$. The popular class of "Actor-critic" methods explicitly represents (and updates) policy and values.
\item Essentially, replace $\sum_{t' = t}^{T-1} r_{t'} - b(s_t)$ \textcolor{orange}{[Vanilla MC]} 
    with Q-values ($Q(s_t, a_t; \mathbf{w}) - b(s_t)$) \textcolor{orange}{[This is essentially TD]} 
    \\or advantage function $\hat{A}^{\pi}(s_t, a_t)$ where $A^{\pi}(s, a) = Q^{\pi}(s, a) - V^{\pi}(s)$.
\end{itemize}

\begin{thmbox}
    \subsubsection*{Alternative Targets to MC Estimators}
    With vanilla MC, the gradient is estimated by
    \begin{equation*}
        \nabla_{\theta} V(\theta) \approx \frac{1}{m} \sum_{i = 1}^{m} \sum_{t = 0}^{T-1} \textcolor{blue}{R_{t}^{i}} \nabla_{\theta} \log \pi_{\theta} (a_t^{(i)} | s_t^{(i)})
    \end{equation*}

    \textbf{Proposed replacements:}
    \begin{enumerate}
    \item N-step estimators:
        \begin{itemize}
        \item $\hat{R}_t^{(1)} = r_t + \gamma V(s_{t+1})$, $\hat{R}_t^{(2)} = r_t + \gamma r_{t+1} + \gamma^2 V(s_{t+2})$ 
        \item $\hat{R}_t^{(\infty)} = r_t + \gamma r_{t+1} + \gamma r_{t+2} + \dots$
        \end{itemize}
    \item Advantage estimators, by \textit{\textcolor{orange}{subtracting baselines of $V(s_t)$} from above}
        \begin{itemize}
        \item $\hat{A}_t^{(1)} = \hat{R}_t^{(1)} \textcolor{orange}{- V(s_t)} = r_t + \gamma V(s_{t+1}) \textcolor{orange}{- V(s_t)}$ \quad \quad 
            \quad \quad \textcolor{blue}{(\textit{low variance, high bias})}
        \item $\hat{R}_t^{(\infty)} \textcolor{orange}{- V(s_t)} = r_t + \gamma r_{t+1} + \gamma r_{t+2} + \dots \textcolor{orange}{- V(s_t)}$
            \quad \quad \textcolor{blue}{(\textit{high variance, low bias})}
        \end{itemize}
    \end{enumerate}
\end{thmbox}

\begin{hintbox}
    What do the words "actor" and "critic" mean?
    \begin{prfbox}
        \begin{itemize}
        \item "actor": Policy $\pi_{\theta}$ selecting actions
        \item "critic": VFA (say, trained via temporal-difference) to "criticize" actions chosen by actor 
        \item Proposed $Q^{\pi}(s, a)$ or $A^{\pi}(s, a)$ serve as baselines/ targets for actor updates to reduce variance.
        \end{itemize}
    \end{prfbox}
\end{hintbox}

\section{PPO and Its Two Variants}
    \subsection{Problem: Poor Sample Efficiency of PG}
    A PG algo should minimize \# iterations to reach a good (probably suboptimal) policy within time. The limitations of vanilla PG:
    \subsubsection{Poor sample efficiency}
    Variance reduces slowly (even after improvements), because PG is an \textbf{on-policy} expectation: Data immediately \textcolor{red}{discarded} after just one gradient step
    \begin{itemize}
    \item Collect sample estimates from trajectories of same policy (more stable), or \textcolor{magenta}{other policies (off-policy, less stable)}.
    \item Opportunity: Can we take \textcolor{blue}{multiple gradient steps from old data} before new policy?
    \end{itemize}

    \begin{defbox}
        \subsubsection*{Problems of Determining Gradient Step}
        Problem: Difficult to handle step size (dist. in parameter space $\neq$ dist. in policy space)
        \begin{itemize}
        \item e.g. Matrices in tabular case $\Pi = \{ \pi: \pi \in \mathbb{R}^{|S| \times |A|}, \sum_{a} \pi_{s_a} = 1, \pi_{s_a} \geq 0 \}$
            \\VS steps of policy gradient in parameter space $\implies$ \textcolor{magenta}{unable to map/ gauge size}!
        \item SGD of $\theta_{k+1} = \theta_{k} + \alpha_{k} \hat{g}_k$ is subject to \textcolor{red}{performance collapse} with large steps!
            \\e.g. logistic function: small $\Delta \theta$ leads to big policy changes
        \end{itemize}
    \end{defbox}

    \subsection{Policy Performance Bounds}
    The solution to restrict policy change from more than intended is through \textcolor{blue}{policy performance bounds}:
    \begin{defbox}
        \subsubsection*{Distance in Value to Policy}
        To respect distance mapping in policy space, exploit relationships between policy performance:
        \begin{equation*}
            J(\pi') - J(\pi) = \mathbb{E}_{\tau \sim \pi'} \left[ \sum_{t = 0}^{\infty} \gamma^{t} A^{\pi}(s_t, a_t) \right]
            = \frac{1}{1-\gamma} \mathbb{E}_{s \sim d^{\pi'}, a \sim \pi'} \left[ A^{\pi} (s, a) \right]
        \end{equation*}
        Here $d^{\pi}(s) = (1 - \gamma) \sum_{t = 0}^{\infty} \gamma^{t} P(s_t = s | \pi)$ is the \textbf{weighted} distribution of states. Making use,

        Now, rewrite the objective:
        $\begin{aligned}
            \mathop{\max}\limits_{\pi'} J(\pi') & = \mathop{\max}\limits_{\pi'} J(\pi') - J(\pi) & = \mathop{\max}\limits_{\pi'} \mathbb{E}_{\tau \sim \pi'} \left[ \sum_{t = 0}^{\infty} \gamma^{t} A^{\pi}(s_t, a_t) \right] \\
            & & = \mathop{\max}\limits_{\pi'} \frac{1}{1-\gamma} \mathbb{E}_{s \sim d^{\pi'}, a \sim \pi'} \left[ A^{\pi} (s, a) \right] \\ 
            & & = \mathop{\max}\limits_{\pi'} \frac{1}{1-\gamma} \mathbb{E}_{s \sim d^{\pi'}, \textcolor{red}{a \sim \pi}} \left[ \textcolor{red}{\frac{\pi' (a | s)}{\pi(a | s)}} A^{\pi} (s, a) \right]
        \end{aligned}$
    \end{defbox}

    \textbf{Why rewrite?}
    \begin{enumerate}
    \item Now, performance of $\pi'$ is defined in advantages from $\pi$.
    \item Requires trajectories sampled from $\pi'$ (desired: from $\pi$ because our tweak features \textcolor{red}{$s \sim d^{\pi}$}).
    \end{enumerate}

    We have a useful approximation:
    \begin{thmbox}
        \subsubsection*{Relative Policy Performance Bounds}
        \begin{equation*}
            J(\pi') - J(\pi) \approx \mathbb{L}_{\pi} (\pi') \quad\quad \text{for close $\pi'$ and $\pi$ ($d^{\pi'} = d^{\pi}$)}
        \end{equation*}
        Approximation quality is ensured by relative policy performance bounds:
        \begin{equation*}
            |J(\pi') - \left( J(\pi) + \mathbb{L}_{\pi} (\pi') \right)| \leq C \sqrt{\mathbb{E}_{s \sim d^{\pi}} \left[ D_{KL} (\pi' || \pi) [s]\right]}
        \end{equation*}
    \end{thmbox}
    
    But what is $\mathbb{E}_{s \sim d^{\pi}} \left[ D_{KL} (\pi' || \pi) [s]\right]$? 
    \begin{defbox}
        \subsubsection*{KL-Divergence}
        Such divergence measures distance between \textcolor{blue}{probability distributions}:
        \begin{equation*}
            D_{KL} (P || Q) = \sum_{x} P(x) \log \frac{P(x)}{Q(x)}
        \end{equation*}
        KL satisfies $D_{KL}(P || P) = 0, D_{KL}(P || Q) \geq 0, D_{KL}(P || Q) \neq D_{KL}(Q || P)$.

        Between policies, $D_{KL}(\pi' || \pi) [s] = \sum_{a \in A} \pi' (a | s) \log \frac{\pi' (a | s)}{\pi (a | s)}$
    \end{defbox}

    After the approximation, we can optimize using trajectories \textcolor{blue}{sampled from the old policy $\pi$}!
    \begin{thmbox}
        \subsubsection*{Policy Optimization under Bounded KL Approximation}
        Policy improvement can be estimated by sampling from \textcolor{red}{old policy $\pi$}!
        \begin{equation*}
            J(\pi') - J(\pi) \approx L_{\pi}(\pi') 
            = \mathbb{E}_{\textcolor{red}{\tau \sim \pi}} \left[ \sum_{t = 0}^{\infty} \gamma^{t} \frac{\pi'(a_t, s_t)}{\pi(a_t, s_t)} A^{\pi}(s_t, a_t) \right]
        \end{equation*}
    \end{thmbox}

\subsection{Variants of PPO}
Taking advantage of policy performance bounds, PPO penalizes large policy change iterations. There are two methods: 
\begin{enumerate}
\item regularization term on KL-divergence (Adaptive Penalty)
\item pessimistic objective on far-away policies (Clipped Objective)
\end{enumerate}

\begin{thmbox}
    \subsubsection*{Adaptive Penalty}
    \begin{equation*}
        \theta_{k+1} = \mathop{\arg\max}\limits_{\theta} L_{\theta_k} (\theta) - \beta \bar{D}_{KL} (\theta || \theta_k)
    \end{equation*}
    Here, KL-divergence is an expectation:
    \begin{equation*}
        \bar{D}_{KL} (\theta || \theta_k) = \mathbb{E}_{s \ sim d^{\pi_k}} D_{KL} \left( \theta_{k} (\cdot | s), \pi_{\theta} (\cdot | s) \right)
    \end{equation*}
    Penalty coefficient $\beta_k$ changes between iterations:
    \begin{itemize}
    \item Initiate policy param $\theta_0$, initial KL penalty $\beta_0$, target KL-divergence $\delta$
    \item Compute policy update (iterate $\theta$) by K steps of minibatch SGD (via Adam)
    \item Control KL-divergence to be around $\delta$ by adjusting penalty:
        \\If $\bar{D}_{KL} (\theta || \theta_k) \geq 1.5 \delta$ then $\beta_{k+1} = 2 \beta_{k}$;
        \\elif $\bar{D}_{KL} (\theta || \theta_k) \leq \frac{\delta}{1.5}$ then $\beta_{k+1} = \frac{1}{2} \beta_{k}$
    \end{itemize}
    This KL penalty is called "adaptive" because of \textcolor{orange}{how $\beta_k$ changes (adapts quickly)} according to KL-divergence.
\end{thmbox}

An alternative approach is clipping: restrict via \textcolor{blue}{pessimistically treating objective value far away from $\theta_k$}.
\begin{thmbox}
    \subsubsection*{Clipped Objective on Policy Changes}
    \begin{itemize}
    \item Define relative probability change: $r_{t}(\theta) = \frac{\pi_{\theta} (a_t | s_t)}{\pi_{\theta_k} (a_t | s_t)}$ 
    \item The new objective is
        \begin{equation*}
            L_{\theta_{k}}^{\text{Clip}} (\theta) 
            = \mathbb{E}_{\tau \sim \pi_{k}} \left[ \sum_{t = 0}^{T} \left[ \textcolor{magenta}{\min} \left( r_{t} (\theta) \hat{A}_t^{\pi_{k}}, \textcolor{blue}{\text{clip}(r_{t}(\theta), \textcolor{orange}{1 - \epsilon, 1 + \epsilon})} \hat{A}_t^{\pi_{k}} \right) \right] \right]
        \end{equation*}
        In other words, $r_{t} (\theta)$ is \textcolor{blue}{clipped between ($\textcolor{orange}{1 - \epsilon, 1 + \epsilon}$)}; hyperparameter $\epsilon$ usually set at 0.2.
    \item Policy update: $\theta_{k+1} = \mathop{\arg\max} ~~~ \limits_{\theta} L_{\theta_{k}}^{\text{Clip}} (\theta)$
    \end{itemize}
\end{thmbox}

Here, Clipping discentivizes going far from $\theta_{k+1}$:
\begin{itemize}
\item Left graph below: when the advantage function $A > 0$; right graph when $A < 0$.
\item $L_{\theta_{k}}^{\text{Clip}} (\theta)$'s increase is suppressed at extreme values towards the sign of $A$.
\item Clipping is simple to implement but works well compared to KL penalty.
\end{itemize}

\begin{figure}
    \includegraphics[width=0.8\textwidth]{lec6-clipped-values.png}
\end{figure}

PPO's performance consistently \textcolor{blue}{tops other algos, hence wildly popular.}
\begin{itemize}
\item Today, it is a key component of ChatGPT (readings: \href{https://openai.com/index/openai-baselines-ppo/}{OpenAI blog (2017)}, \href{https://arxiv.org/pdf/1707.06347}{Publication by Schulman et. al. (2017)})
\item Different outcomes with reward scaling/ learning rate annealing.
\end{itemize}

\begin{hintbox}
    Why KL-divergence applied is an expectation?
    \begin{prfbox}
        \begin{itemize}
        \item Recall that the two distributions (of the same variable) occurs in probabilities.
        \item To compare, take average (expectation) over states (discounted distribution). 
        \end{itemize}
    \end{prfbox}
    How does the two methods compare?
    \begin{prfbox}
        \textbf{KL Penalty}
        \begin{itemize}
        \item Idea: Tune penalty $\beta$ adaptively to control KL around the target threshold
        \item Pros: Conceptually direct (literally penalize KL)
        \item Cons: Tricky to schedule/ tune/ stabilize $\beta$
        \end{itemize}

        \textbf{Clipped Objective}
        \begin{itemize}
            \item Idea: Force objective $r_{t} \theta$ to plateau when significantly deviate from 1.
            \item Pros: Simple to implement and good practical performance
            \item Cons: Less direct than penalty/ constraint
        \end{itemize}
        In practice, \textcolor{magenta}{clipping} is much more common/ popular given \textcolor{magenta}{implementation and performance}.
    \end{prfbox}
\end{hintbox}

\end{document}