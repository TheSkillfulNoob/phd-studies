\documentclass{article}
\usepackage{minted}
\input{preamble-eng}
\title{Stanford CS234: Lectures 1 to 3}
\begin{document}

\section{Introduction}

\subsection{Overview}
\subsubsection{On General Machine Learning}
ML has seen a huge (exponential) increase in academic interest since 2000. It is categorized as follows:
\begin{enumerate}
    \item Supervised and unsupervised Learning (SL/ UL): Only involves generalization
    $\implies$ Differs in whether there is labelling
    \item Imitation Learning (IL): behaviour cloning, assuming input demonstrations of GOOD policies
    \item \textbf{Reinforcement Learning (RL): model-based learning (i.e. given reward info, states reached and actions taken)}
\end{enumerate}

\subsubsection{On Reinforcement Learning}
RL refers to \textbf{\textcolor{blue}{Sequential decision making}} to make good decisions under uncertainty.
\begin{defbox}
  \subsubsection*{Steps of RL}
  RL involves the following four steps:
  \begin{enumerate}
  \item \textbf{Optimization}:
      Find optimal decisions yielding best/ very good outcomes (e.g. minimum distance inter-city route).
  \item \textbf{Delayed consequences}:
      Plan and reason long term ramifications (e.g. saving for retirement), through temporal credit assignment (i.e. identifying which past actions led to current rewards?).
  \item \textbf{Exploration}:
      Learn/ explore by making decisions (that impacts/ reinforces what we learn about from rewards, e.g. riding and falling from a bike).
  \item \textbf{Generalization}:
      Policy maps (= compresses) past experience to action.
  \end{enumerate}
\end{defbox}

\begin{hintbox}
    \subsubsection*{Why not always choose the best-known action in decision making?}
    New actions might lead to even better rewards (exploration-exploitation trade-off).
    \begin{prfbox}
        Example: Trying a new restaurant vs. Going to favorite one?
    \end{prfbox}
\end{hintbox}

\subsubsection{Course Flow}
\begin{itemize}
\item High-level learning goals:
    Understand theoretical and empirical approaches for evaluating reinforcement learning algorithm quality
\item Flow: 
    First explore MDP $\implies$ model-free (policy evaluation and control) $\implies$ function approximation $\implies$ policy search + exploration
\end{itemize}

\subsection{Basic Layout of RL}
RL algorithms involve \textbf{State, actions, reward model and dynamics model}:
\begin{itemize}
\item RL aims to select sequence of actions to maximize E[future reward],
\item hence involves balancing immediate and long term rewards.
\end{itemize}
Example: Choose sequence of web advertisement to maximize view time

\begin{defbox}
  \subsubsection*{Mathematical Formulation of RL algorithms}
  For each time $t$, a reinforcement learning agent (A) that interacts with the world (W) would...
  \begin{enumerate}
  \item A takes \textcolor{blue}{action $a_t$},
  \item W updates $a_t$, emits \textcolor{blue}{observation $o_t$} and \textcolor{blue}{reward $r_t$}
  \item A receives \textcolor{blue}{history $h_t$} containing {$a_i$, $o_i$, $r_i$} for all time $i$, cumulative history is called state $s_t$.
  \end{enumerate}
\end{defbox}

In such sequential dynamic programming problems of RL, we have the following considerations:
\begin{enumerate}
\item \textbf{States}: Is the state Markov? Is the world partially observable?
\item \textbf{Dynamics}: Are dynamics deterministic or stochastic?
\item \textbf{Horizon of effect}: is future states taken into account?
\end{enumerate}

\subsubsection{States: Markov Assumption}
In many cases, we have the Markov assumption:
\begin{thmbox}
  \subsubsection*{Markov Assumption: Definition}
  The future is independent of past given present (i.e. \textcolor{blue}{state} is a sufficient statistic of \textcolor{cyan}{history}):
  \begin{equation*}
    P(s_{t+1} | \textcolor{blue}{s_t}, a_t ) = P(s_{t+1} | \textcolor{cyan}{h_t}, a_t )
  \end{equation*}
\end{thmbox}

\subsubsection{Dynamics: Model, Policy, Value Function}
RL Algos must contain $\ge1$ of {Model, Policy, Value Function}:

\subsubsection*{Model}
A MDP (Markov Decision Process) contains
\begin{enumerate}
\item \textbf{Transitional dynamics model}, predicting next agent state
    $P(s_{t+1} = s' | s_t = s, a_t = a)$
\item \textbf{Reward model}, predicting immediate reward 
    $r(s_t = s | a_t = a) = E[ r_t | s_t = s, a_t = a]$
\end{enumerate}

\begin{hintbox}
    \subsubsection*{Why Markov?}
    \begin{prfbox}
        \begin{itemize}
            \item Simple, as long as history is in part of state (in practice, assume $s_t$ = $o_t$)
            \item $s_t$ affects computational complexity, data required and result performance.
        \end{itemize}
    \end{prfbox}
    \subsubsection*{Why do we start with MDP?}
    \begin{prfbox}
        \textbf{Structured framework} to model decision-making in uncertain environments.
        \\ $\implies$ Easier to apply RL algorithms, defining \textcolor{magenta}{states, actions, rewards, and transitions}.
    \end{prfbox}
\end{hintbox}

\subsubsection*{Policy}
Policy (denoted by $\pi$) determines how actions are chosen (i.e. $\pi: S → A$). It can be
\begin{enumerate}
\item \textbf{Determinstic}: $\pi(s) = a$, or
\item \textbf{Stochastic}: $\pi(a | s) = P(a_t = a |  s_t = s)$
\end{enumerate}

\subsubsection*{Value Function}
Value function (denoted by $V^{\pi}$) discounts sum of future rewards under policy $\pi$, to quantify goodness of states/ actions:
\begin{equation*}
  V^{\pi}(s_t = s) = \mathbb{E}_{\pi} \left[ r_t + \gamma r_{t+1} + \gamma^2 r_{t+2} + \dots | s_t = s \right]
\end{equation*}

\begin{defbox}
  \subsubsection*{Discounting Factor Considerations}
  Discount factor $\gamma$ weighs immediate vs future rewards:
  \begin{itemize}
  \item $\gamma = 0$: only cares about immediate;
  \item $\gamma = 1$: possible for finite episode length.
  \item To prevent infinite returns, conveniently set $\gamma$ < 1.
  \end{itemize}
\end{defbox}

\subsection{Agent and Feedback: MRP}
RL agents can be either
\begin{enumerate}
\item \textbf{Model-based}: explicit model, may (not) have policy and value functions;
\item \textbf{Model-free}: no model, but value and/or polict functions must be explicit.
\end{enumerate}

\begin{defbox}
  \subsubsection*{Learning Mechanisms}
  Given a model of world (dynamics and reward), of finite set of states and actions.
  \\Two actions for the RL agent include:
  \begin{enumerate}
  \item \textbf{Evaluation}: Estimate/ predict expected rewards given policy.
  \item \textbf{Control}: find the best policy via optimization.
  \end{enumerate}
\end{defbox}

Now consider the learning algorithm under Markov Chain (Markov Process):
\begin{itemize}
\item Markov processes involve sequences of random states with Markov property
    \\They are defined by (finite) set of states $S$ and transition model $P$, specifying $P(s_{t+1} = s' | s_t = s)$
\item In a transition model, there is \textcolor{blue}{no reward}, hence no actions are defined.
    \\For finite ($N$) states, express P as $N \times N$ matrix.
\item To bridge the stochasticity in quantifying rewards/ value, 
    Markov Reward Processes (MRP) and Markov Decision Processes (MDP) are used.
\end{itemize}

\subsubsection{Markov Reward Processes (MRP)}
MRP \textcolor{blue}{includes the reward} into consideration.
\begin{itemize}
\item It requires states $\mathbf{S}$, transition model $\mathbf{P}_{n \times n}$, reward $R(s_t = s)$, and discount factor $\gamma \in [0, 1]$. 
\item With H steps in each episode (can be infinite), the return
    \begin{equation*}
      G_t = r_t + \gamma r_{t+1} + … + \gamma^{H-1} r_{t+H-1}
    \end{equation*} 
\item Define the state value $V(s) = E[G_t | S_t = s] = \textcolor{cyan}{R(s)} + \textcolor{orange}{\gamma \sum_{s' \in S} \mathbb{P}(s' | s) V(s')}$.
    \\\textcolor{cyan}{The first term} is the immediate reward; the \textcolor{orange}{second term} is the discounted sum of future rewards.
\end{itemize}

Obviously we are interested in computing the value of a MRP. There are two methods:
\begin{thmbox}
    \subsubsection*{Analytical and Iterative Solutions of MRP}
    \textcolor{orange}{Method 1 (Matrix inverse)} involves writing in matrix-vector forms:
    \begin{itemize}
    \item If the number of states $n$ is finite, rewards $\mathbf{R}$ is a vector.
    \item Express $V(s)$ in the MRP using a matrix equation
        \begin{equation*}
          \mathbf{V}_{n \times 1} = \mathbf{R}_{n \times 1} + \gamma \mathbf{P}_{n \times n} \mathbf{V}_{n \times 1}
        \end{equation*}
    \item Rearranging, the solution $\mathbf{V} = (\mathbf{I} - \gamma \mathbf{P})^{-1} \mathbf{R}$.
        \\Direct solve takes around $O(n^3)$ time.
    \end{itemize}

    \vspace{2em}
    \textcolor{orange}{Method 2 (Iterative Solution)} involves looping until convergence:
    \begin{itemize}
    \item Dynamic programming by initializing $V_0(s) = 0$ for all $s$
    \item For each iteration $k$, and all states $s \in S$: \hspace{4em} $V_{k}(s) = \textcolor{cyan}{R(s)} + \textcolor{orange}{\gamma \sum_{s' \in S} \mathbb{P}(s' | s) V_{k-1}(s')}$
    \item Each iteration takes $O(n^2)$ time (because $n = |S|$).
    \end{itemize}
  \end{thmbox}

\begin{hintbox}
    \subsubsection*{Why $\mathbf{I} - \gamma \mathbf{P}$ is Invertible?}
    \begin{prfbox}
        With $\gamma < 1$ (ensuring convergence), and the fact that $\mathbf{P}$ is stochastic (rows sum to 1):
        \begin{itemize}
        \item The eigenvalues of $P$ are at most 1.
        \item When scaled by $\gamma < 1$, largest eigenvalue $\lambda_1 < \gamma = 1$. 
        \\This ensures $\mathbf{I} - \gamma \mathbf{P}$ is full rank (diagonally dominant).
        \end{itemize}
    \end{prfbox}
\end{hintbox}

\section{How to make good decisions given MDP?}
\subsection{Solving MDP by reducing to MRP}
\begin{defbox}
    \subsubsection*{MDP as MRP Expansion}
    \begin{itemize}
    \item MRP($S$, $P$, $R$, $\gamma$) + Actions $A$ = MDP($S$, $A$, $P$, $R$, $\gamma$)
        \begin{prfbox}
            Example: Consider a robot vacuum moving around a room:
            \begin{itemize}
            \item Only models probabilities of movement without considering actions: MRP
            \item Includes actions like 'turn left' or 'move forward: MDP
            \end{itemize}
        \end{prfbox}
    \item MDP($S$, $A$, $P$, $R$, $\gamma$) + Policy $\pi(a | s)$ = MRP($S$, $P^{\pi}$, $R^{\pi}$, $\gamma$)
        \\ Use same technique of computing MRP to evaluate MDP policy!
    \item Various settings of MDPs:
        \begin{enumerate}
        \item \textbf{Single-state}: Bandits
        \item \textbf{Continuous states}: Optimal control
        \item \textbf{If state is history}: Partially observable MDPs (POMDPs)
        \end{enumerate}
    \end{itemize}
\end{defbox}

We can solve MDP as an MRP as follows:
\begin{prfbox}
    \subsubsection*{Solving MDP Iteratively}
    For $k = 1$ till convergence, for any $s \in S$, the state value function $V$ is iterated with:
    \begin{center}
        $V_k^{π}(s) = \textcolor{orange}{\sum_{a} \pi(a | s)} \textcolor{cyan}{[r (s | a) + \gamma \sum_{s' \in S} P(s' | s,a) V_{k-1}^\pi(s')]}$ 
        \quad\quad where $s' \in S$
    \end{center}
    \textit{This is called a Bellman backup equation.}
    \begin{itemize}
    \item Current value at $k$th iteration is obtained by 
            \textcolor{orange}{summing the policy probabilities} varying state $s$, 
            \textcolor{cyan}{weighted by the value} (current reward + future rewards discounted by one step).
    \item For deterministic $\pi(s)$, i.e. $P(\pi(s) | s) = 1$, the formula reduces to
    \begin{center}
        $V_k^{π}(s) = r (s | \pi(s)) + \gamma \sum_{s' \in S} P(s' | s, \pi(s)) V_{k-1}^\pi(s')$
    \end{center}
    \end{itemize}
\end{prfbox}
\begin{hintbox}
    Where does Bellman Equation arise?
    \begin{prfbox}
        It arises naturally when breaking down long-term decision problems (by recursion) into smaller subproblems, as in policy evaluation.
    \end{prfbox}
\end{hintbox}

\subsection{How to find the optimal policy under MDP?}
\begin{itemize}
\item Given $|A|$ distinct options for each of the $|S|$ states, there are a total of $|A|^{|S|}$ deterministic policies possible.
\item Optimal policy may NOT be unique (but is deterministic and stationary (does not depend on time step) for MDP in infinite horizon!), but there exists a \textbf{unique optimal value function}.
\item Such optimal policy can be found by "controlling", i.e. optimizing $\pi(s) = \mathop{\arg\min}\limits_{\pi} V_{\pi}(s)$
\end{itemize}

\begin{hintbox}
    \subsubsection*{Why is such optimal policy deterministic?}
    \begin{prfbox}
        \begin{itemize}
        \item Randomness in an MDP \textcolor{magenta}{comes from state transitions}, not from the policy itself.
        \item Given enough iterations, an optimal policy learns \textcolor{magenta}{best action in every state} $implies$ randomness is unnecessary.
        \end{itemize}
    \end{prfbox}
\end{hintbox}

There are primarily \textbf{THREE methods} in finding the optimal policy under MDP:
\subsubsection{Policy iteration (PI)}
PI is more efficient than enumeration. Its process is as follows:
\begin{thmbox}
  \subsubsection*{PI Process}
  \begin{enumerate}
  \item [(1)] \textbf{Initialization}: Initialize $\pi_0(s)$ randomly for all states $s$.
  \item [(2)] \textbf{Check L1 Norm in Loop}: Iterate when \textcolor{blue}{\texttt{i == 0}} OR \textcolor{blue}{\texttt{$|| \pi_i - \pi_{i-1} ||_{1} > 0$}},
        \\i.e. \textit{if policy changes for any state}.
        \\Each time, compute  $V_\pi(i)$ (policy evaluation), and $\pi_{i+1}$ (policy improvement) to iterate new \texttt{i}
  \end{enumerate}
\end{thmbox}

But why does PI guarantee generating better policies?
\begin{itemize}
\item Define the \textbf{state action value} (Q-function):
    \begin{equation*}
        Q_{\pi} (s, a) = R(s, a) + \gamma \sum_{s' \in S} P(s' | s, a) V_{\pi} (s')
    \end{equation*}
    \\It differs from $V_{\pi}$ that \textcolor{blue}{action $a$ is taken in addition} to following the policy.
\item Then for new policy $\pi_{i+1}$, 
      \begin{center}
          $\pi_{i+1} (s) = \mathop{\arg\min}\limits_{a} Q^{\pi_i}{\pi}(s, a) \forall s \in S$
      \end{center}
      \\is a deterministic quantity.
\item To improve the state-action value by varying $a$,
      \begin{center}
          $\mathop{\max}\limits_{a} Q^{\pi_i} (s, a) \geq
          Q^{\pi_i} (s, \pi_{i}(s)) = R(s, \pi(s)) + \gamma \sum_{s' \in S} P(s' | s, \pi_i(s)) V^{{\pi}_i} (s')
          = V^{{\pi}_i} (s)
          $
      \end{center}
\end{itemize}

This is more rigorously defined as follows:
\begin{thmbox}
    \subsubsection*{Monotonic Improvement in Policy}
    Definition: $V^{\pi_{1}} \geq V^{\pi_{2}}$ if $V^{\pi_{1}}(s) \geq V^{\pi_{2}}(s)$ for all states $s$.
    (\textit{Strict inequality $V^{\pi_{i+1}} > V^{\pi_{i}}$  if $\pi_{i}$ is suboptimal.})
    \begin{prfbox}
        \textbf{Implication}: By taking \textcolor{blue}{$\pi_{i+1} (s)$ for one action} and following $\pi_{i+1}$ forever, expected sum of rewards is \textcolor{blue}{at least as good} as always following $\pi_{i}$.
        Since we always pick a better action, policies never get worse!
    \end{prfbox}
\end{thmbox}

\begin{hintbox}
    \subsubsection*{Regarding change (and iteration) of policies in PI}
    \begin{prfbox}
        \begin{itemize}
            \item If policy doesn't change, it cannot be changed anymore.
            \item There is a maximum number of policy iterations (since a finite number of options for each finite state).
        \end{itemize}
    \end{prfbox}
    \subsubsection*{Why state action value is defined? Why store Q instead of V?}
    \begin{prfbox}
        \begin{itemize}
            \item It evaluates the expected return \textcolor{magenta}{taking action $a$ a first}, then act optimally afterward.
            \item Instead of storing $V(s)$, store $Q(s, \textcolor{magenta}{a})$ to \textcolor{magenta}{directly optimize} actions.
        \end{itemize}
    \end{prfbox}
\end{hintbox}

\subsubsection{Value iteration (VI)}
The idea is to maintain \textbf{optimal value} of starting in state $s$, if finite number of steps $k$ left in episode.
\\Then, iterate to consider \textcolor{blue}{longer and longer episodes}:
\begin{defbox}
    \subsubsection*{Bellman Operator}
    For a value function, the Bellman Operator $B$ returns a new value that \textcolor{blue}{improves if possible}:
    \begin{center}
        $BV(s) = \textcolor{blue}{\mathop{\max}\limits_{a}} \left[\textcolor{blue}{R(s, a)} + \gamma \sum_{s' \in S} \textcolor{blue}{P(s' | s, a)} V(s')\right]$
    \end{center}
    For a policy,
    \begin{center}
        $B^{\pi} V(s) = R^{\pi} (s) + \gamma \sum_{s' \in S} P^{\pi} (s' | s) V(s')$
    \end{center}
    \begin{expbox}
        Example: Updating est. travel time in Google Maps as new traffic data arrives (i.e. iteratively refines value estimates).
    \end{expbox}
\end{defbox}

\begin{hintbox}
    \subsubsection*{How to "maintain optimal value"? Why does $a$ disappear?}
    \begin{prfbox}
        \begin{itemize}
            \item Maintaining optimal value means ensuring that for every state, compute best expected return by \textcolor{magenta}{considering all possible actions}. 
            \item As we only keep the \textcolor{magenta}{highest value across actions}, we remove $a$.
            \item We only care about values during iteration: Extract optimal policy by checking which action led to that value
        \end{itemize}
    \end{prfbox}
\end{hintbox}

\begin{thmbox}
    \subsubsection*{VI Process}
    \begin{enumerate}
    \item [(1)] \textbf{Initialization}: $k = 1$, $V_0(s) = 0$ for all states $s$.
    \item [(2)] \textbf{Check Norm in Loop}: Loop until convergence (i.e. checking $L_{\infty} = ||V_{k+1} - V_{k}||_{\infty} \leq \epsilon$)
          \\For each state $s$ in $S$, iterate $V_{k+1}(s) = \mathop{\max}\limits_{a} \left[ R(s, a) + \gamma \sum_{s' \in S} P(s' | s, a) V_{k}(s') \right]$
          \\This iteration step is equivalently $V_{k+1} = BV_k$
    \end{enumerate}
    For the optimal policy for \textcolor{orange}{finite horizon $H$}, iterate (loop) from $i = 0:H$ (exclusive of $H$):
    \begin{itemize}
    \item $V_{k+1}(s) = \mathop{\max}\limits_{a} \left[ R(s, a) + \gamma \sum_{s' \in S} P(s' | s, a) V_{k} (s') \right]$
    \item $\pi_{k+1}(s) = \mathop{\arg \max}\limits_{a} \left[ R(s, a) + \gamma \sum_{s' \in S} P(s' | s, a) V_{k} (s') \right]$
    \end{itemize}
\end{thmbox}

But why does VI guarantee convergence?
\begin{itemize}
\item Quantify convergence with \textcolor{blue}{contraction operator $O$}, satisfying
    \begin{center}
        $||OV - OV'||_n \leq ||V - V'||_n$ for any $L_n$ norm.
    \end{center}
\item Converges when $\gamma < 1$ OR ending up in terminal state with probability 1!
\item VI converges to \textbf{unique solution} for discrete state and action spaces
\item Policy evaluation = compute \textbf{fixed point} of $B_{\pi}$, by repeatedly applying operator until V stops changing, i.e.
    \begin{center}
        $V^{\pi} = B^{\pi} B^{\pi} … B^{\pi} V$
    \end{center}
\end{itemize}

\begin{hintbox}
    What does “fixed point” here mean?
    \begin{prfbox}
        \begin{itemize}
            \item Applying the operator doesn't change the result anymore.
            \item Signifies convergence (e.g. in PI, determine whether the policy should be updated).
        \end{itemize}
    \end{prfbox}
    Does the initialization of values in value iteration impact anything?
    \begin{prfbox}
        \begin{itemize}
            \item It only affects convergence speed, but same final result. 
            \item Because Bellman backup is a contraction mapping.
        \end{itemize}
    \end{prfbox}
\end{hintbox}

\subsubsection*{A brief summary of PI vs VI}
\begin{center}
    \begin{tabular}{|c||c|c|}
    \hline
     Type & PI & VI \\ \hline
     To compute... & 
        \textcolor{blue}{Infinite} horizon value &
        Optimal \textcolor{blue}{finite $k$-horizon} value 
    \\ \hline
    Use & 
        Select another (better) policy &
        Find optimal policy, by incrementing time horizon 
    \\ \hline
        Convergence & 
        \#Iterations $\leq$ Enumeration of all deterministic policies, &
        Depends on discounting factor $\gamma$ , 
    \\ & i.e. of size $|A|^{|S|}$ & possible to exceed $|A|^{|S|}$
    \\ \hline
    \end{tabular}
\end{center}

\subsection{Monte Carlo (MC) Methods for MDP}
Given dynamics and reward models, can do policy evaluation through DP.
\begin{center}
    $V_k^{\pi}(s) = r (s | \pi(s)) + \textcolor{orange}{\gamma \sum_{s' \in S} P(s' | s, \pi(s)) V_{k-1}^{\pi}(s')}$
\end{center}

Here, $V_k^{\pi}$ is an estimate of $V^{\pi}$, and \textcolor{orange}{the second part} can be replaced by \textbf{bootstrapping} if dynamics/ reward models are unknown but deterministic 
: when there is \textcolor{blue}{no access to true MDP}.
\begin{itemize}
\item MC follows policy $\pi$ to \textcolor{blue}{generate sample trajectories} and take expectation on sample return to approximate (estimate) $V^{\pi}(s)$.
\item Rationale: Mean return converges to value (by LLN)
    \\\textit{Note: All trajectories may not be of the same length because of MDP.}
\item Requires \textcolor{blue}{episodic settings (terminal state/ finite horizon)} to end episode
    \\ $\implies$ Terminate episode before averaging.
\end{itemize}

\begin{hintbox}
    \subsubsection*{Advantages of MC}
    \begin{itemize}
    \item Weak assumption (No need Markov assumption/ knowing MDP dynamics and rewards)
    \item Sometimes preferred over DP for policy evaluation even with known dynamics model and reward. (Explain why?)
    \end{itemize}
\end{hintbox}

The process and variations of MC is as follows:
\begin{thmbox}
    \subsubsection*{MC Process}
    \begin{enumerate}
    \item \textbf{Initialize}:
        Number of samples $N(s) = 0$, grand total $G(s) = 0$ for each state $s \in S$.
    \item \textbf{Loop}:
        \begin{itemize}
        \item For each episode $i$, sample trajectories $(s_{i,1}, a_{i,1}, r_{i,1}), (s_{i,2}, a_{i,2}, r_{i,2}), \dots , s_{i, t_i}$
        \item Compute return value $G_{i, t}$ as return from time step $t$:
            \begin{center}
                $G_{i, t} = r_{i, t} + \gamma r_{i, t+1} + \dots + \gamma^{T_i-t} r_{i,T_i}$
            \end{center}
        \item Update state values for each time step t until end of episode i:
            \\Increment \# visits: $N(s) = N(s) + 1$
            \\Increment total return: $G(s) = G(s) + G_{i, t}$
            \\Update estimate $V^{\pi}(s) = G(s)/ N(s)$
        \end{itemize}
    \end{enumerate}  
\end{thmbox}

There are \textbf{Three variations} for MC:
\begin{defbox}
    \subsubsection*{Variations of MC}
    \begin{enumerate}
        \item [(i)] \textcolor{blue}{First-time MC}: Increment $N(s)$ for the \textcolor{blue}{first time visiting state $s$} in episode $i$, i.e. $N(s) = 1$
        \item [(ii)] \textcolor{blue}{Every-time MC}: Increment $N(s)$ \textcolor{blue}{each time visiting state $s$} in episode $i$
        \item [(iii)] \textcolor{blue}{Incremental MC}: After obtaining the new return value $G_{i, t}$, rewrite the update from $V^{\pi}(s)$ to $V^{\pi}(s)'$:
            \begin{prfbox}
                \begin{aligned}
                    V^{\pi}(s)' & = \frac{G(s)'}{N(s)'} = \frac{G(s) + G_{i, t}}{N(s) + 1} = \frac{\textcolor{cyan}{G(s)}}{N(s) + 1} + \frac{G_{i, t}}{N(s) + 1} \\
                    & = \textcolor{cyan}{V^{\pi}(s)} \times \frac{\textcolor{cyan}{N(s)}}{N(s) + 1} + \frac{G_{i, t}}{N(s) + 1} \\
                    & = \textcolor{cyan}{V^{\pi}(s)} + \textcolor{magenta}{\frac{1}{N(s) + 1}} \left( G_{i, t} - V^{\pi}(s) \right) \\
                    V^{\pi}(s)' & = \textcolor{cyan}{V^{\pi}(s)} + \textcolor{magenta}{\alpha} \left( G_{i, t} - V^{\pi}(s) \right)
                \end{aligned}
            \end{prfbox}
    \end{enumerate}
\end{defbox}

The general form replaces the learning rate \textcolor{magenta}{$\alpha = 1/[N(s) + 1]$} to other rates. Viewed from this perspective:
\begin{center}
    \begin{tabular}{|c||c|c|c|}
    \hline
     MC Type & Every Visit & First Visit & General incremental form \\ \hline
     $\alpha$ & 
        $\frac{1}{N (s_{i, t})}$ &
        1 if $N(s) = 0$, 0 otherwise &
        Variable
    \\ \hline
    Properties & 
        Biased, but consistent and offers better MSE &
        Unbiased estimator of $V^{\pi}(s)$ by LLN & 
        Rate larger than $1/N(s_{i,t})$ is helpful in non-stationary domains
    \\ \hline
    \end{tabular}
\end{center}

\begin{hintbox}
    Q: What is an intuitive explanation to such every-visit "bias"? What does it mean by helpful in non-stationary domains?
\end{hintbox}

\subsection{Evaluating Policy Estimation Quality}
We take theoretical and computational aspects for consideration:
\begin{itemize}
\item Theoretical: Statistical efficiency, consistency and empirical accuracy (MSE) 
\item Computation: Runtime and Memory Complexity
\end{itemize}

\begin{defbox}
    \subsubsection{Bias and Variance}
    The empirical accuracy of learning models is quantified by Mean Squared Error (MSE):
    \begin{center}
        $\text{MSE}(\hat{\theta}) = \text{Var}(\hat{\theta}) + \text{Bias} (\hat{\theta})^2$
    \end{center}
    \begin{itemize}
    \item $\text{Var}(\hat{\theta}) = \mathbb{E}_{x | \theta} [(\hat{\theta} - \mathbb{E}[\hat{\theta}])^2]$
    \item $\text{Bias}(\hat{\theta})^2 = (\mathbb{E}_{x | \theta}[\hat{\theta}] - \theta)^2$
    \end{itemize}
\end{defbox}

\textbf{Consistency} quantifies how reliable an estimator is:
\begin{itemize}
\item If there are $n$ data points of $x$, $\lim_{n \to \infty} \mathbb{P}(|\hat{\theta}_n - \theta| > \epsilon) = 0$ for arbitrary $\epsilon$.
\item In other words, estimates (produced by the estimator) "converge" to the true parameter value.
\end{itemize}

\begin{hintbox}
    An unbiased estimator may not be consistent. Two simple examples:
    \begin{prfbox}
    \begin{itemize}
    \item The variance estimator $\frac{1}{n} \sum (y_i - y_n)^2$ is biased but consistent.
    \item The data point $X_1$ as an estimator of $\mu$ for $N(\mu, \sigma^2)$ is unbiased but NOT consistent.
    \end{itemize}
    \end{prfbox}
\end{hintbox}

\subsubsection{How does MC fare under those criteria?}
\textbf{(1) On Convergence:}
For MC, under varying learning rate $\alpha$, incremental MC will converge to true value of policy, $V^{\pi}(s_j)$ if 
\begin{enumerate}
\item $\sum_{n = 1}^{\infty} \alpha_n (s_j) = \infty$, but
\item $\sum_{n = 1}^{\infty} \alpha_n^{2} (s_j) < \infty$.
\end{enumerate}

\textbf{(2) On MSE:}
MC yields a \textcolor{blue}{high-variance estimator} that takes a lot of data to reduce.

\section{Value Function Approximation}
Recall in optimizing policy $\pi$, we can either \textbf{evaluate} (compute $Q^{\pi}$) or \textbf{improve} (update $\pi$ given $Q^{\pi}$).

Problem: Deterministic policies in approximation yield \textcolor{red}{no exploration!}
\begin{itemize}
\item If $\pi$ deterministic, $Q(s, a)$ is undefined for $a \neq \pi(s)$. Can't learn about actions without trying them!
\item Exploitation of past experience that proved high reward. 
\end{itemize}

Solution: Assign probability on unvisited pathways!
\begin{itemize}
\item $\epsilon$-greedy pathways selects argmax action with \textcolor{blue}{$1-\epsilon$ probability}, else random. i.e.
    $\pi(a | s) =$ 
    \left\{
    \begin{aligned}
        &\mathop{\arg\max}\limits_{a} Q(s, a), & ~\text{probability}~~ \textcolor{blue}{1 - \epsilon} + \frac{\epsilon}{|A|} \\
        &a' \neq \mathop{\arg\max}\limits_{a} Q(s, a), & ~\text{probability}~~ \frac{\epsilon}{|A|} \\
    \end{aligned}
    \right.
\item $\epsilon$-greedy policy guarantees monotonic improvement, and can estimate $Q(s, a) \forall s, a$.
\end{itemize}

\subsubsection{Greedy in Limit of Infinite Exploration (GLIE)}
\begin{defbox}
    GLIE: If $\lim_{i \to \infty} N_i(s, a) \to \infty$ (i.e. all state-action pairs visited infinite times)
    \begin{itemize}
    \item Behavior policy (policy used to act in the world) converges to greedy policy
    \item GLIE is $\epsilon$-greedy if $\epsilon = \frac{1}{i} \to 0$.
    \item MC under GLIE converges to optimal value, i.e. $Q(s, a) \to Q^{*}(s, a)$
    \end{itemize}
\end{defbox}

\begin{hintbox}
    Q: What is the convergence to optimal $Q^{*}$ function and the expected performance of MC control with $\epsilon$-greedy policy?
    \\When may MC online control fail to compute the optimal $Q^{*}$?
\end{hintbox}

Other than on-policy learning (through direct experience following the policy), \textbf{off-policy} learning exists (experience gathered from \textcolor{blue}{following a different policy}).
\\In particular, Q-learning is a popular area of off-policy learning:
\begin{defbox}
    \subsubsection*{Q-learning Definition and Properties}
    \begin{itemize}
        \item Q-learning estimates $Q^{\pi^{*}}$ by acting with another $\pi_b$ by maintaining $Q$ estimates and bootstrapping.
        \item Update step: 
            \begin{equation*}
                Q(s_t, a_t) = Q(s_t, a_t) + \alpha \left( [r_t + \gamma \textcolor{blue}{\mathop{\arg\max}\limits_{a'} Q(s_{t+1}, a')} - Q(s_t, a_t)] \right)
            \end{equation*}
            where $\textcolor{blue}{\mathop{\arg\max}\limits_{a'} Q(s_{t+1}, a')}$ is by following another behaviour policy.
        \item Take $a_t \sim \pi_b(s_t)$: Sample action from $\epsilon$-greedy policy. Here $\pi_b$ is $\epsilon$-greedy w.r.t. $Q$.
        \item It builds results through stochastic approximation, decreasing step sizes to satisfy Bellman contraction property, and utilizing bounded rewards and value function.
    \end{itemize}
\end{defbox}

Q-learning for finite-state and finite-action MDPs converges to optimal value $Q^{*}(s, a)$ given:
\begin{itemize}
\item Policy sequence $\pi_t(a|s)$ satisfies conditions of GLIE.
\item $\sum_{t = 1}^{\infty} \alpha_t = \infty$ and $\sum_{t = 1}^{\infty} \alpha_t^2 < \infty$
\end{itemize}

\subsection{Function Approximation}
Function \textbf{approximation} carries the following two purposes:
\begin{enumerate}
\item Avoid \textcolor{blue}{explicit storing/ learning} for every single state and action (model/ value/ policy)
\item \textcolor{blue}{Compact representation}, generalized across states and actions (i.e. less memory and computation needed)
\end{enumerate}
There are \textbf{THREE} forms of approximation: \textcolor{blue}{bootstrapping, sampling, and VFA}.

\textcolor{magenta}{With a model}, we carry out VFA by assuming we can "query $(s, a)$" to return $Q^{\pi}(s, a)$, i.e. supervised learning:
\begin{defbox}
    \subsubsection*{Model-based VFA: Process}
    \begin{itemize}
    \item \textbf{Objective}: \textbf{approximate representation} of $Q^{\pi}$ given \textcolor{magenta}{parameterized form $\hat{Q}(s, a; \mathbf{w})$}
    \item \textbf{Input}: $\mathbf{w}$ is a parameter vector of weights.
    \item \textbf{Method}: Minimize loss function between $Q^{\pi}(s, a)$ and approximation $\hat{Q}(s, a; \mathbf{w})$:
        \begin{equation*}
            J(\mathbf{w}) = \mathbb{E}_{\pi}[ \left(Q^{\pi} (s, a) - \hat{Q}(s, a; \mathbf{w}) \right)^2]
        \end{equation*}
    \item \textbf{Means}: Find local minimum with gradient descent
        \begin{equation*}
            \delta \mathbf{w} = -\frac{1}{2} \alpha \textcolor{orange}{\nabla_{\mathbf{w}} J(\mathbf{w})}
        \end{equation*}
    \item \textbf{Improvement}: \textcolor{blue}{Stochastic} gradient descent (SGD) uses finite (often one) sample to \textcolor{blue}{approximate gradient}
        \begin{equation*}
            \textcolor{orange}{\nabla_{\mathbf{w}} J(\mathbf{w})} = -2 \mathbb{E}_{\pi}[ Q^{\pi} (s, a) - \hat{Q}(s, a; \mathbf{w})] \nabla_{\mathbf{w}} \hat{Q}
        \end{equation*}
    \end{itemize}
\end{defbox} 

\begin{hintbox}
    How does SGD differ from GD? What is "finite \# samples"?
    [See answer from GPT]
\end{hintbox}

\subsubsection{VFA without Model}
\textcolor{magenta}{Without a model (oracle)}, follow policy $\pi$/ prior data to estimate $V^{\pi}$ and $Q^{\pi}$.
\begin{defbox}
    \subsubsection*{Model-free VFA: Process}
    \begin{itemize}
    \item \textbf{Input}: Generated sample paths by following fixed policy $\pi$.
    \item \textbf{Method}: Through lookup table storing estimates of $V^{\pi}$ or $Q^{\pi}$
        \\$\to$ Update estimates after each episode (MC)/ step (TD)
    \item Change the estimate update step to include fitting \textcolor{blue}{function approximator}.
        \\\textit{How?} ~~ Return $G_t$ is unbiased but noisy sample of true $Q^{\pi}(s_t, a_t)$
        \\$\implies$Do supervised learning on $<(s_i, a_i), G_i>$.
    \end{itemize}
\end{defbox}

\subsection{VFA under TDL}
Recall that TDL uses \textcolor{blue}{bootstrapping and sampling} to approximate $V^{\pi}$:
\begin{equation*}
    V^{\pi}(s) = V^{\pi}(s) + \alpha \left( \textcolor{orange}{r + \gamma V^{\pi}(s')} - V^{\pi}(s) \right)
\end{equation*}
where the temporal target $\textcolor{orange}{r + \gamma V^{\pi}(s')}$ is \textbf{biased} against $V^{\pi}(s)$.
\begin{itemize}
\item When with a look table, represent each state value with a separate table entry.
\item in VFA, target is $\textcolor{orange}{r + \gamma V^{\pi}(s' \textbf{; \mathbf{w}})}$, also \textbf{biased} against $V^{\pi}(s)$.
\item Equivalent to supervised learning on $<(s_i, r_i), \textcolor{orange}{\gamma \hat{V}^{\pi} (s_{i+1}; \mathbf{w})}>$ and finding weights $\mathbf{w}$ to minimize MSE with SGD:
\begin{equation*}
    J(\mathbf{w}) = \mathbb{E}_{\pi}[ \left( \textcolor{orange}{r_j + \gamma V^{\pi} (s_{j+1}, \mathbf{w})} - \hat{V}(s_j; \mathbf{w}) \right)^2]
\end{equation*}
\end{itemize}

\begin{hintbox}
    Question: Why the function involve $V$ here but not $Q$ like in model-based VFA?
\end{hintbox}

\subsubsection{Control using VFA}
Intuitively the few main points are:
\begin{itemize}
\item \textbf{Estimation}: $Q^{\pi}(s, a; \mathbf{w}) \approx Q^{\pi}$ by \textbf{sampling} gradients from SGD, for local minimum. 
\item \textbf{Interleaving}: \textcolor{blue}{Perform $\epsilon$-greedy policy improvement} between VFA approximations.
\end{itemize}

There are a few incremental approaches to determine the increment $\Delta \mathbf{w}$ for action-value function approximation:
\begin{enumerate}
\item \textbf{General form}: use \textcolor{cyan}{target value} 
    \begin{center}
        $\Delta \mathbf{w} = \alpha( \textcolor{cyan}{Q(s_t, a_t)} - \hat{Q}(s_t, a_t; \mathbf{w}) ) \nabla_{\mathbf{w}} \hat{Q}(s_t, a_t; \mathbf{w})$ 
    \end{center}
\item \textbf{MC Simulation}: use \textcolor{cyan}{return $G_t$} as substitute target
    \begin{center}
        $\Delta \mathbf{w} = \alpha( \textcolor{cyan}{G_t} - \hat{Q}(s_t, a_t; \mathbf{w}) ) \nabla_{\mathbf{w}} \hat{Q}(s_t, a_t; \mathbf{w})$ 
    \end{center}
\item \textbf{SARSA}: use \textcolor{cyan}{TD target, i.e. $r + \gamma \hat{Q}(s', a'; \mathbf{w})$} 
    \begin{center}
        $\Delta \mathbf{w} = \alpha( \textcolor{cyan}{r + \gamma \hat{Q}(s', a'; \mathbf{w})} - \hat{Q}(s, a; \mathbf{w}) ) \nabla_{\mathbf{w}} \hat{Q}(s, a; \mathbf{w})$ 
    \end{center}
\item \textbf{Q-Learning}: use \textcolor{cyan}{related TD target} to SARSA
    \begin{center}
        $\Delta \mathbf{w} = \alpha( \textcolor{cyan}{
            r + \gamma \mathop{\arg\max}\limits_{a'} Q(s', a'; \mathbf{w})
        } - \hat{Q}(s, a; \mathbf{w}) ) \nabla_{\mathbf{w}} \hat{Q}(s, a; \mathbf{w})$ 
    \end{center}
\end{enumerate}

\begin{hintbox}
    Q: What is the original purpose of interleaving? How does it cause problems?
    Q: Recall, how is the target value in general form VFA increment determined?
    Q: Recall why is there $t$ in MC/ General form for $s$ and $a$ but NOT SARSA/ Q-learning?
    Q: Recall: What are the major differences between SARSA and Q-learning by tweaking the TD target?
\end{hintbox}

\subsection{Deadly Triad and Remedies}
\begin{hintbox}
    "Bellman operators are contractions, but value function approximation fitting can be an expansion". What does this mean?
\end{hintbox}
\end{document}