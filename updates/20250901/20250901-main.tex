\documentclass[xcolor=dvipsnames]{beamer}
\usepackage{
    amsmath, % mathematical eq and symbols
    graphicx, % manipulate graphics (images)
    xcolor, % color options
    colortbl, % colours to tables
    soul, % highlighting
    multirow, % merging cells in tables
    bm, % bolding variables
    arydshln, % dashed lines
    subcaption, % 2 figures on the same row
    textcomp, gensymb, % degree symbol
    amsfonts, % real number bold
    booktabs, % formal tables
    scrextend, % block Indents
    blkarray, % transitionary probability matrices
    tikz, pgfplots, % coordinate/ fig plots
    ifthen, % logic statements
    CJKutf8, % Chinese characters
    hyperref,
    wrapfig, % For positioning figures
    changepage,
    etoolbox % To adjust minted inside Beamer
}

\geometry{paperwidth=11.5cm,paperheight=11.5cm}

\newcommand\highlight[2][]{\tikz[overlay]\node[fill=cyan,inner sep=2pt, anchor=text, rectangle, rounded corners=1mm,#1] {#2};\phantom{#2}}

\renewcommand\label[1]{}
\pgfplotsset{compat = 1.18}
\usetikzlibrary{matrix}
\tikzstyle{startstop} = [rectangle, rounded corners, minimum width=3cm, minimum height=1cm,text centered, draw=black, fill=red!30]
\tikzstyle{process} = [rectangle, minimum width=3cm, minimum height=1cm, text centered, draw=black, fill=blue!30]
\tikzstyle{arrow} = [thick,->,>=stealth]

% Colors
\definecolor{customtableheaderblue}{HTML}{AAAAEE}
\definecolor{customblue}{RGB}{224, 255, 255}
\definecolor{custompink}{RGB}{255, 204, 229}
\definecolor{customgreen}{RGB}{224, 255, 239}
\definecolor{custompurple}{RGB}{239, 224, 255}

\newcolumntype{M}[1]{>{\centering\arraybackslash}m{#1}}

\usepackage[most]{tcolorbox}
\usepackage{minted}
\BeforeBeginEnvironment{minted}{\vspace{0em}} % Adjust spacing before and after minted
\AfterEndEnvironment{minted}{\vspace{-1em}}

\usetikzlibrary{shapes.geometric, arrows}
\tcbuselibrary{theorems}

\newtcolorbox[auto counter]{explanation}[1][]{
title={\bfseries Explanation},
enhanced,
drop shadow={black!50!white},
boxed title style = {
top = 0.1in,
bottom = 0.1in,
size = small,
colback = custompink,
},
coltitle=black,
attach boxed title to top left = {
xshift=-\tcboxedtitlewidth/8,
yshift=-\tcboxedtitleheight/2
},
top = 0.15in,
colback = customgreen,
#1}

\newtcolorbox[auto counter]{summary}[1][]{
title={\bfseries Summary},
enhanced,
drop shadow={black!50!white},
boxed title style = {
top = 0.1in,
bottom = 0.1in,
size = small,
colback = custompink,
},
coltitle=black,
attach boxed title to top left = {
xshift=-\tcboxedtitlewidth/6,
yshift=-\tcboxedtitleheight/2
},
top = 0.15in,
colback = customblue,
#1}

\newtcolorbox[auto counter, number within=subsection]{question}[1][]{
title={\bfseries Question},
enhanced,
drop shadow={black!50!white},
boxed title style = {
top = 0.1in,
bottom = 0.1in,
size = small,
colback = custompink,
},
coltitle=black,
attach boxed title to top left = {
xshift=-\tcboxedtitlewidth/10,
yshift=-\tcboxedtitleheight/2
},
top = 0.15in,
colback = custompurple,
#1}

\usetheme{Frankfurt}
\newcommand{\ThemeVariationOne}{%
  \usecolortheme[named=MidnightBlue]{structure}
}

\newcommand{\ThemeVariationTwo}{%
  \usecolortheme{crane}
  \setbeamercolor{section in head/foot}{fg=white, bg=violet}
  \setbeamercolor{item projected}{fg=white, bg=purple}
}

\newcommand{\ThemeVariationThree}{%
  \usecolortheme{crane}
  \definecolor{mint}{HTML}{98FB98}
  \definecolor{amazon}{HTML}{387B54}
  \definecolor{seafoam}{HTML}{80F9AD}
  \definecolor{forest}{HTML}{0B6623}
  \definecolor{pistachio}{HTML}{93C572}
  \setbeamercolor{title}{fg=white, bg=pistachio}
  \setbeamercolor{section in head/foot}{fg=white, bg=forest}
  \setbeamercolor{frametitle}{fg=white, bg=pistachio}
  \setbeamercolor{item projected}{fg=white, bg=forest}
  \setbeamercolor{item}{fg=black}
}

\newcommand{\ThemeVariationFour}{%
  \usecolortheme{wolverine}
  \setbeamercolor{section in head/foot}{fg=black}
  \setbeamercolor{background canvas}{bg=yellow!15}
}

\usepackage[usenames,dvipsnames]{color}
\newtcolorbox{prfbox}{colback=gray!10,colframe=black!70,boxrule=0pt,arc=0pt,boxsep=2pt,left=2pt,right=2pt,leftrule=0pt}
\newtcolorbox{thmbox}{colback=orange!25,colframe=orange!85,boxrule=0pt,arc=0pt,boxsep=2pt,left=2pt,right=2pt,leftrule=2.5pt}
\newtcolorbox{defbox}{colback=blue!5,colframe=blue!70,boxrule=0pt,arc=0pt,boxsep=2pt,left=2pt,right=2pt,leftrule=2.5pt}
\newtcolorbox{expbox}{colback=green!10,colframe=green!70,boxrule=0pt,arc=0pt,boxsep=2pt,left=2pt,right=2pt,leftrule=2.5pt}
\newtcolorbox{warnbox}{colback=red!15,colframe=red!70,boxrule=0pt,arc=0pt,boxsep=2pt,left=2pt,right=2pt,leftrule=2.5pt}
\newtcolorbox{hintbox}{colback=violet!10,colframe=violet!70,boxrule=0pt,arc=0pt,boxsep=2pt,left=2pt,right=2pt,leftrule=2.5pt}

% Command to include the title
\newcommand{\customtitle}[2]{
    \title{#1}
    \date{#2}
    \begin{frame}
        \titlepage
        \tableofcontents
    \end{frame}
}

\author{
    Andrew LAM
}

\setlength{\parindent}{0pt}
\renewcommand{\baselinestretch}{1.15}
\geometry{papersize={14cm,10.5cm}}
\ThemeVariationOne

\begin{document}
\customtitle
{Starting Things Off!}
{1 Sept, 2025}

\begin{frame}{Motivation of Updating This Way}
    \begin{itemize}
    \item \textcolor{blue}{Systematic organization of information}, knowledge, paper-reading and writing ideas for the upcoming years 
    \item Keep progress/ moves visible by \textcolor{blue}{discretizing at reasonable steps}
    \item \textcolor{blue}{Make related communication smoother}/ predictable by keeping people around updated
    \end{itemize}
\end{frame}

\begin{frame}{One Slide Introduction: Myself}
    \begin{itemize}
    \item \textbf{Name}: Andrew LAM (\begin{CJK*}{UTF8}{bsmi}林碩風\end{CJK*})
    \item \textbf{Undergrad major}: HKUST QFIN + MATH (2020-24)
    \item \textbf{Motivation of picking IEDA}: 
        "Quantifying the World" with research, 
        after quantifying in finance under a quant trading internship
    \item \textbf{Hobbies}:
        Cycling, Tetris (original proposed topic), Arcade, Weekly Reflections
    \end{itemize}
\end{frame}

\begin{frame}{What Have I Done: Jun-Aug}
    I focused on three main arcs, to properly \textcolor{blue}{recharge and explore myself} before the
    research seasons kick in:
    \begin{enumerate}
    \item \textbf{NLP Applications}
        \begin{itemize}
        \item Inspired by my \textcolor{blue}{Buffett Letters' Parts of Speech} project (June 2025)
        \item Developing project for \textcolor{blue}{analyzing dating markets'} posting sentiments 
        \end{itemize}
    \item \textbf{Math Education}
        \begin{itemize}
        \item Constructed a \textcolor{blue}{repository supporting by-topic classifications} of 
            various high school level math exams (HKDSE/ HKALE/ Camb STEP...) and automatic document 
            generation under Python infrastructure ($\sim$6K Questions, Jul-Aug 2025)
        \item Will spend a minor chunk of time in AY2025-26 on \textcolor{blue}{supervising pro-bono math mock} events 
            for HKDSE students, after gathering a team of volunteers
        \end{itemize}
    \item \textbf{Personal: Arcade and Fitness}
        \begin{itemize}
        \item Developed \textcolor{blue}{Bowling hobby} and maintained weekly sessions (Jun-Aug)
        \item Spent 100+ hours on \textcolor{blue}{rhythm game} \textit{Maimai} to train musical timing, hand 
            muscles and explore songs of different genres. 
        \end{itemize}
    \end{enumerate}
\end{frame}

\begin{frame}{Regarding Research - Reading}
    Prof. Zhang assigned 
    \textcolor{cyan}{\href{https://papers.ssrn.com/sol3/papers.cfm?abstract_id=4262186}{this research project's extension}} to me in June:
    \begin{itemize}
    \item \textbf{Context}: Multi-echelon inventory management (MEIM)
    \item \textbf{Paper Insights}: applies Multi-Agent Deep Reinforcement Learning (MADRL), 
        specifically Heterogeneous-Agent Proximal Policy Optimization (HAPPO),
        by modelling the system as a Partially Observable Markov Game (POMG)
    \item \textbf{Results}: Agents trained via HAPPO surpass single-agent RL and heuristic baselines, 
        achieving lower total costs and mitigating the bullwhip effect through upfront-only information sharing
    \end{itemize}

    \begin{defbox}
        \textbf{Question from Prof:}
        \\What if each agent is only interested in maximizing its own benefit without concerning the system’s benefit:
        Would cooperation still arise?
    \end{defbox}
\end{frame}

\begin{frame}{Regarding Research - Rough Plan I}
    A rough sketch of my follow-ups in the next few weeks:
    \begin{itemize}
    \item \textcolor{blue}{Do reading on MEIM's background, common complexities} 
        (lead times, demand uncertainty, and inter-echelon coordination),
        and \textcolor{blue}{existing models} such as 
        \begin{enumerate}
        \item Traditional inventory Base Stock and (s, S); 
        \item Dual-Sourcing;
        \item employing heterogeneous agents to address decentralized decision-making)
        \end{enumerate}
    \item \textcolor{blue}{Sensitivity and Initialization} of inventory problems (Initial inventories/backlogs, stochastic demand, 
        sensitivity analysis on lead times and costs)
    \end{itemize}
\end{frame}

\begin{frame}{Regarding Research - Rough Plan II}
    \small
    Summarize \textcolor{blue}{Problem Variations and Literature insights}, on:
    \begin{enumerate}
    \item \textcolor{blue}{Bullwhip Effect} (Demand variability increases from downstream to upstream due to information distortion)
    \item \textcolor{blue}{Supply Uncertainty}: Price fluctuations, lead-time variability, shipping losses
    \item \textcolor{blue}{Information-Sharing Scenarios}: Fully centralized, decentralized, and hybrid (like CTDE)
    \end{enumerate}
    Plan to read several other relevant papers (e.g. Lee et al. (1997) on bullwhip effect/ 
    Clark and Scarf (1960) and Shang and Song (2003) for base-stock policy optimality and heuristics)   

    \vspace{1em}
    After going through the above, I will work on the article's HAPPO Implementation via Python:
    \begin{itemize}
    \item Set up actors' \textcolor{blue}{observation spaces} with inventories, backlogs, and order flows.
    \item \textcolor{blue}{Define neural network models} (e.g., using PyTorch with GRU units).
    \item Custom \textcolor{blue}{simulation environments} for episodic training.
    \item \textcolor{blue}{Implement PPO} via stable-baselines or custom methods.
    \end{itemize}
\end{frame}

\begin{frame}{Upcoming Academic Year: Plans}
    To balance learning (input) and research (output), I intend to take 12 credits:
    \begin{itemize}
    \item \textbf{Required}: ENGG 6780, PDEV 6770B, IEDA 6800C
    \item \textbf{Dept}: IEDA 5270 (Engg Statistic \& Data Analytic)
    \item \textbf{Relevant to research}: MATH 5411 (Adv Probability Theory I)
    \item \textbf{Language}: LANG 5005 (Communicating Research in English)
    \item \textbf{For Interest}: COMP 5711 (Introduction to Advanced Algorithmic Techniques)
    \end{itemize}
\end{frame}

\begin{frame}{Any Questions/ Advice?}
    Not gonna make this sounding too formal :P
\end{frame}

\end{document}